% Generated by Sphinx.
\def\sphinxdocclass{report}
\documentclass[a4paper,8pt,english]{sphinxmanual}


\usepackage{cmap}
\usepackage[T1]{fontenc}
\usepackage{amsfonts}
\usepackage{babel}
\usepackage{times}
\usepackage[Bjarne]{fncychap}
\usepackage{longtable}
\usepackage{sphinx}
\usepackage{multirow}
\usepackage{eqparbox}


\addto\captionsenglish{\renewcommand{\figurename}{Fig. }}
\addto\captionsenglish{\renewcommand{\tablename}{Table }}
\SetupFloatingEnvironment{literal-block}{name=Listing }


	% Use some font with UTF-8 support with XeLaTeX
        \usepackage{fontspec}
        \setsansfont{DejaVu Serif}
        \setromanfont{DejaVu Sans}
        \setmonofont{DejaVu Sans Mono}

     \usepackage[margin=0.5in, top=1in, bottom=1in]{geometry}
        \usepackage{ifthen}

        % Put notes in color and let them be inside a table
	\definecolor{NoteColor}{RGB}{204,255,255}
	\definecolor{WarningColor}{RGB}{255,204,204}
	\definecolor{AttentionColor}{RGB}{255,255,204}
	\definecolor{ImportantColor}{RGB}{192,255,204}
	\definecolor{OtherColor}{RGB}{204,204,204}
        \newlength{\mynoticelength}
        \makeatletter\newenvironment{coloredbox}[1]{%
	   \setlength{\fboxrule}{1pt}
	   \setlength{\fboxsep}{7pt}
	   \setlength{\mynoticelength}{\linewidth}
	   \addtolength{\mynoticelength}{-2\fboxsep}
	   \addtolength{\mynoticelength}{-2\fboxrule}
           \begin{lrbox}{\@tempboxa}\begin{minipage}{\mynoticelength}}{\end{minipage}\end{lrbox}%
	   \ifthenelse%
	      {\equal{\py@noticetype}{note}}%
	      {\colorbox{NoteColor}{\usebox{\@tempboxa}}}%
	      {%
	         \ifthenelse%
	         {\equal{\py@noticetype}{warning}}%
	         {\colorbox{WarningColor}{\usebox{\@tempboxa}}}%
		 {%
	            \ifthenelse%
	            {\equal{\py@noticetype}{attention}}%
	            {\colorbox{AttentionColor}{\usebox{\@tempboxa}}}%
		    {%
	               \ifthenelse%
	               {\equal{\py@noticetype}{important}}%
	               {\colorbox{ImportantColor}{\usebox{\@tempboxa}}}%
	               {\colorbox{OtherColor}{\usebox{\@tempboxa}}}%
		    }%
		 }%
	      }%
        }\makeatother

        \makeatletter
        \renewenvironment{notice}[2]{%
          \def\py@noticetype{#1}
          \begin{coloredbox}{#1}
          \bf\it
          \par\strong{#2}
          \csname py@noticestart@#1\endcsname
        }
	{
          \csname py@noticeend@\py@noticetype\endcsname
          \end{coloredbox}
        }
	\makeatother

     

\title{SuperH architecture implementation manual}
\date{March 08, 2018}
\release{4.16.0-rc4+}
\author{The kernel development community}
\newcommand{\sphinxlogo}{}
\renewcommand{\releasename}{Release}

\makeindex

\makeatletter
\def\PYG@reset{\let\PYG@it=\relax \let\PYG@bf=\relax%
    \let\PYG@ul=\relax \let\PYG@tc=\relax%
    \let\PYG@bc=\relax \let\PYG@ff=\relax}
\def\PYG@tok#1{\csname PYG@tok@#1\endcsname}
\def\PYG@toks#1+{\ifx\relax#1\empty\else%
    \PYG@tok{#1}\expandafter\PYG@toks\fi}
\def\PYG@do#1{\PYG@bc{\PYG@tc{\PYG@ul{%
    \PYG@it{\PYG@bf{\PYG@ff{#1}}}}}}}
\def\PYG#1#2{\PYG@reset\PYG@toks#1+\relax+\PYG@do{#2}}

\expandafter\def\csname PYG@tok@gd\endcsname{\def\PYG@tc##1{\textcolor[rgb]{0.63,0.00,0.00}{##1}}}
\expandafter\def\csname PYG@tok@gu\endcsname{\let\PYG@bf=\textbf\def\PYG@tc##1{\textcolor[rgb]{0.50,0.00,0.50}{##1}}}
\expandafter\def\csname PYG@tok@gt\endcsname{\def\PYG@tc##1{\textcolor[rgb]{0.00,0.27,0.87}{##1}}}
\expandafter\def\csname PYG@tok@gs\endcsname{\let\PYG@bf=\textbf}
\expandafter\def\csname PYG@tok@gr\endcsname{\def\PYG@tc##1{\textcolor[rgb]{1.00,0.00,0.00}{##1}}}
\expandafter\def\csname PYG@tok@cm\endcsname{\let\PYG@it=\textit\def\PYG@tc##1{\textcolor[rgb]{0.25,0.50,0.56}{##1}}}
\expandafter\def\csname PYG@tok@vg\endcsname{\def\PYG@tc##1{\textcolor[rgb]{0.73,0.38,0.84}{##1}}}
\expandafter\def\csname PYG@tok@vi\endcsname{\def\PYG@tc##1{\textcolor[rgb]{0.73,0.38,0.84}{##1}}}
\expandafter\def\csname PYG@tok@mh\endcsname{\def\PYG@tc##1{\textcolor[rgb]{0.13,0.50,0.31}{##1}}}
\expandafter\def\csname PYG@tok@cs\endcsname{\def\PYG@tc##1{\textcolor[rgb]{0.25,0.50,0.56}{##1}}\def\PYG@bc##1{\setlength{\fboxsep}{0pt}\colorbox[rgb]{1.00,0.94,0.94}{\strut ##1}}}
\expandafter\def\csname PYG@tok@ge\endcsname{\let\PYG@it=\textit}
\expandafter\def\csname PYG@tok@vc\endcsname{\def\PYG@tc##1{\textcolor[rgb]{0.73,0.38,0.84}{##1}}}
\expandafter\def\csname PYG@tok@il\endcsname{\def\PYG@tc##1{\textcolor[rgb]{0.13,0.50,0.31}{##1}}}
\expandafter\def\csname PYG@tok@go\endcsname{\def\PYG@tc##1{\textcolor[rgb]{0.20,0.20,0.20}{##1}}}
\expandafter\def\csname PYG@tok@cp\endcsname{\def\PYG@tc##1{\textcolor[rgb]{0.00,0.44,0.13}{##1}}}
\expandafter\def\csname PYG@tok@gi\endcsname{\def\PYG@tc##1{\textcolor[rgb]{0.00,0.63,0.00}{##1}}}
\expandafter\def\csname PYG@tok@gh\endcsname{\let\PYG@bf=\textbf\def\PYG@tc##1{\textcolor[rgb]{0.00,0.00,0.50}{##1}}}
\expandafter\def\csname PYG@tok@ni\endcsname{\let\PYG@bf=\textbf\def\PYG@tc##1{\textcolor[rgb]{0.84,0.33,0.22}{##1}}}
\expandafter\def\csname PYG@tok@nl\endcsname{\let\PYG@bf=\textbf\def\PYG@tc##1{\textcolor[rgb]{0.00,0.13,0.44}{##1}}}
\expandafter\def\csname PYG@tok@nn\endcsname{\let\PYG@bf=\textbf\def\PYG@tc##1{\textcolor[rgb]{0.05,0.52,0.71}{##1}}}
\expandafter\def\csname PYG@tok@no\endcsname{\def\PYG@tc##1{\textcolor[rgb]{0.38,0.68,0.84}{##1}}}
\expandafter\def\csname PYG@tok@na\endcsname{\def\PYG@tc##1{\textcolor[rgb]{0.25,0.44,0.63}{##1}}}
\expandafter\def\csname PYG@tok@nb\endcsname{\def\PYG@tc##1{\textcolor[rgb]{0.00,0.44,0.13}{##1}}}
\expandafter\def\csname PYG@tok@nc\endcsname{\let\PYG@bf=\textbf\def\PYG@tc##1{\textcolor[rgb]{0.05,0.52,0.71}{##1}}}
\expandafter\def\csname PYG@tok@nd\endcsname{\let\PYG@bf=\textbf\def\PYG@tc##1{\textcolor[rgb]{0.33,0.33,0.33}{##1}}}
\expandafter\def\csname PYG@tok@ne\endcsname{\def\PYG@tc##1{\textcolor[rgb]{0.00,0.44,0.13}{##1}}}
\expandafter\def\csname PYG@tok@nf\endcsname{\def\PYG@tc##1{\textcolor[rgb]{0.02,0.16,0.49}{##1}}}
\expandafter\def\csname PYG@tok@si\endcsname{\let\PYG@it=\textit\def\PYG@tc##1{\textcolor[rgb]{0.44,0.63,0.82}{##1}}}
\expandafter\def\csname PYG@tok@s2\endcsname{\def\PYG@tc##1{\textcolor[rgb]{0.25,0.44,0.63}{##1}}}
\expandafter\def\csname PYG@tok@nt\endcsname{\let\PYG@bf=\textbf\def\PYG@tc##1{\textcolor[rgb]{0.02,0.16,0.45}{##1}}}
\expandafter\def\csname PYG@tok@nv\endcsname{\def\PYG@tc##1{\textcolor[rgb]{0.73,0.38,0.84}{##1}}}
\expandafter\def\csname PYG@tok@s1\endcsname{\def\PYG@tc##1{\textcolor[rgb]{0.25,0.44,0.63}{##1}}}
\expandafter\def\csname PYG@tok@ch\endcsname{\let\PYG@it=\textit\def\PYG@tc##1{\textcolor[rgb]{0.25,0.50,0.56}{##1}}}
\expandafter\def\csname PYG@tok@m\endcsname{\def\PYG@tc##1{\textcolor[rgb]{0.13,0.50,0.31}{##1}}}
\expandafter\def\csname PYG@tok@gp\endcsname{\let\PYG@bf=\textbf\def\PYG@tc##1{\textcolor[rgb]{0.78,0.36,0.04}{##1}}}
\expandafter\def\csname PYG@tok@sh\endcsname{\def\PYG@tc##1{\textcolor[rgb]{0.25,0.44,0.63}{##1}}}
\expandafter\def\csname PYG@tok@ow\endcsname{\let\PYG@bf=\textbf\def\PYG@tc##1{\textcolor[rgb]{0.00,0.44,0.13}{##1}}}
\expandafter\def\csname PYG@tok@sx\endcsname{\def\PYG@tc##1{\textcolor[rgb]{0.78,0.36,0.04}{##1}}}
\expandafter\def\csname PYG@tok@bp\endcsname{\def\PYG@tc##1{\textcolor[rgb]{0.00,0.44,0.13}{##1}}}
\expandafter\def\csname PYG@tok@c1\endcsname{\let\PYG@it=\textit\def\PYG@tc##1{\textcolor[rgb]{0.25,0.50,0.56}{##1}}}
\expandafter\def\csname PYG@tok@o\endcsname{\def\PYG@tc##1{\textcolor[rgb]{0.40,0.40,0.40}{##1}}}
\expandafter\def\csname PYG@tok@kc\endcsname{\let\PYG@bf=\textbf\def\PYG@tc##1{\textcolor[rgb]{0.00,0.44,0.13}{##1}}}
\expandafter\def\csname PYG@tok@c\endcsname{\let\PYG@it=\textit\def\PYG@tc##1{\textcolor[rgb]{0.25,0.50,0.56}{##1}}}
\expandafter\def\csname PYG@tok@mf\endcsname{\def\PYG@tc##1{\textcolor[rgb]{0.13,0.50,0.31}{##1}}}
\expandafter\def\csname PYG@tok@err\endcsname{\def\PYG@bc##1{\setlength{\fboxsep}{0pt}\fcolorbox[rgb]{1.00,0.00,0.00}{1,1,1}{\strut ##1}}}
\expandafter\def\csname PYG@tok@mb\endcsname{\def\PYG@tc##1{\textcolor[rgb]{0.13,0.50,0.31}{##1}}}
\expandafter\def\csname PYG@tok@ss\endcsname{\def\PYG@tc##1{\textcolor[rgb]{0.32,0.47,0.09}{##1}}}
\expandafter\def\csname PYG@tok@sr\endcsname{\def\PYG@tc##1{\textcolor[rgb]{0.14,0.33,0.53}{##1}}}
\expandafter\def\csname PYG@tok@mo\endcsname{\def\PYG@tc##1{\textcolor[rgb]{0.13,0.50,0.31}{##1}}}
\expandafter\def\csname PYG@tok@kd\endcsname{\let\PYG@bf=\textbf\def\PYG@tc##1{\textcolor[rgb]{0.00,0.44,0.13}{##1}}}
\expandafter\def\csname PYG@tok@mi\endcsname{\def\PYG@tc##1{\textcolor[rgb]{0.13,0.50,0.31}{##1}}}
\expandafter\def\csname PYG@tok@kn\endcsname{\let\PYG@bf=\textbf\def\PYG@tc##1{\textcolor[rgb]{0.00,0.44,0.13}{##1}}}
\expandafter\def\csname PYG@tok@cpf\endcsname{\let\PYG@it=\textit\def\PYG@tc##1{\textcolor[rgb]{0.25,0.50,0.56}{##1}}}
\expandafter\def\csname PYG@tok@kr\endcsname{\let\PYG@bf=\textbf\def\PYG@tc##1{\textcolor[rgb]{0.00,0.44,0.13}{##1}}}
\expandafter\def\csname PYG@tok@s\endcsname{\def\PYG@tc##1{\textcolor[rgb]{0.25,0.44,0.63}{##1}}}
\expandafter\def\csname PYG@tok@kp\endcsname{\def\PYG@tc##1{\textcolor[rgb]{0.00,0.44,0.13}{##1}}}
\expandafter\def\csname PYG@tok@w\endcsname{\def\PYG@tc##1{\textcolor[rgb]{0.73,0.73,0.73}{##1}}}
\expandafter\def\csname PYG@tok@kt\endcsname{\def\PYG@tc##1{\textcolor[rgb]{0.56,0.13,0.00}{##1}}}
\expandafter\def\csname PYG@tok@sc\endcsname{\def\PYG@tc##1{\textcolor[rgb]{0.25,0.44,0.63}{##1}}}
\expandafter\def\csname PYG@tok@sb\endcsname{\def\PYG@tc##1{\textcolor[rgb]{0.25,0.44,0.63}{##1}}}
\expandafter\def\csname PYG@tok@k\endcsname{\let\PYG@bf=\textbf\def\PYG@tc##1{\textcolor[rgb]{0.00,0.44,0.13}{##1}}}
\expandafter\def\csname PYG@tok@se\endcsname{\let\PYG@bf=\textbf\def\PYG@tc##1{\textcolor[rgb]{0.25,0.44,0.63}{##1}}}
\expandafter\def\csname PYG@tok@sd\endcsname{\let\PYG@it=\textit\def\PYG@tc##1{\textcolor[rgb]{0.25,0.44,0.63}{##1}}}

\def\PYGZbs{\char`\\}
\def\PYGZus{\char`\_}
\def\PYGZob{\char`\{}
\def\PYGZcb{\char`\}}
\def\PYGZca{\char`\^}
\def\PYGZam{\char`\&}
\def\PYGZlt{\char`\<}
\def\PYGZgt{\char`\>}
\def\PYGZsh{\char`\#}
\def\PYGZpc{\char`\%}
\def\PYGZdl{\char`\$}
\def\PYGZhy{\char`\-}
\def\PYGZsq{\char`\'}
\def\PYGZdq{\char`\"}
\def\PYGZti{\char`\~}
% for compatibility with earlier versions
\def\PYGZat{@}
\def\PYGZlb{[}
\def\PYGZrb{]}
\makeatother

\renewcommand\PYGZsq{\textquotesingle}

\begin{document}

\maketitle
\tableofcontents
\phantomsection\label{sh/index::doc}

\begin{quote}\begin{description}
\item[{Author}] \leavevmode
Paul Mundt

\end{description}\end{quote}


\chapter{Memory Management}
\label{sh/index:memory-management}\label{sh/index:superh-interfaces-guide}

\section{SH-4}
\label{sh/index:sh-4}

\subsection{Store Queue API}
\label{sh/index:store-queue-api}\index{sq\_flush\_range (C function)}

\begin{fulllineitems}
\phantomsection\label{sh/index:c.sq_flush_range}\pysiglinewithargsret{void \bfcode{sq\_flush\_range}}{unsigned long\emph{ start}, unsigned int\emph{ len}}{}
Flush (prefetch) a specific SQ range

\end{fulllineitems}


\textbf{Parameters}
\begin{description}
\item[{\code{unsigned long start}}] \leavevmode
the store queue address to start flushing from

\item[{\code{unsigned int len}}] \leavevmode
the length to flush

\end{description}

\textbf{Description}

Flushes the store queue cache from \textbf{start} to \textbf{start} + \textbf{len} in a
linear fashion.
\index{sq\_remap (C function)}

\begin{fulllineitems}
\phantomsection\label{sh/index:c.sq_remap}\pysiglinewithargsret{unsigned long \bfcode{sq\_remap}}{unsigned long\emph{ phys}, unsigned int\emph{ size}, const char *\emph{ name}, pgprot\_t\emph{ prot}}{}
Map a physical address through the Store Queues

\end{fulllineitems}


\textbf{Parameters}
\begin{description}
\item[{\code{unsigned long phys}}] \leavevmode
Physical address of mapping.

\item[{\code{unsigned int size}}] \leavevmode
Length of mapping.

\item[{\code{const char * name}}] \leavevmode
User invoking mapping.

\item[{\code{pgprot\_t prot}}] \leavevmode
Protection bits.

\end{description}

\textbf{Description}

Remaps the physical address \textbf{phys} through the next available store queue
address of \textbf{size} length. \textbf{name} is logged at boot time as well as through
the sysfs interface.
\index{sq\_unmap (C function)}

\begin{fulllineitems}
\phantomsection\label{sh/index:c.sq_unmap}\pysiglinewithargsret{void \bfcode{sq\_unmap}}{unsigned long\emph{ vaddr}}{}
Unmap a Store Queue allocation

\end{fulllineitems}


\textbf{Parameters}
\begin{description}
\item[{\code{unsigned long vaddr}}] \leavevmode
Pre-allocated Store Queue mapping.

\end{description}

\textbf{Description}

Unmaps the store queue allocation \textbf{map} that was previously created by
{\hyperref[sh/index:c.sq_remap]{\emph{\code{sq\_remap()}}}}. Also frees up the pte that was previously inserted into
the kernel page table and discards the UTLB translation.


\section{SH-5}
\label{sh/index:sh-5}

\subsection{TLB Interfaces}
\label{sh/index:tlb-interfaces}\index{sh64\_tlb\_init (C function)}

\begin{fulllineitems}
\phantomsection\label{sh/index:c.sh64_tlb_init}\pysiglinewithargsret{int \bfcode{sh64\_tlb\_init}}{void}{}
Perform initial setup for the DTLB and ITLB.

\end{fulllineitems}


\textbf{Parameters}
\begin{description}
\item[{\code{void}}] \leavevmode
no arguments

\end{description}
\index{sh64\_next\_free\_dtlb\_entry (C function)}

\begin{fulllineitems}
\phantomsection\label{sh/index:c.sh64_next_free_dtlb_entry}\pysiglinewithargsret{unsigned long long \bfcode{sh64\_next\_free\_dtlb\_entry}}{void}{}
Find the next available DTLB entry

\end{fulllineitems}


\textbf{Parameters}
\begin{description}
\item[{\code{void}}] \leavevmode
no arguments

\end{description}
\index{sh64\_get\_wired\_dtlb\_entry (C function)}

\begin{fulllineitems}
\phantomsection\label{sh/index:c.sh64_get_wired_dtlb_entry}\pysiglinewithargsret{unsigned long long \bfcode{sh64\_get\_wired\_dtlb\_entry}}{void}{}
Allocate a wired (locked-in) entry in the DTLB

\end{fulllineitems}


\textbf{Parameters}
\begin{description}
\item[{\code{void}}] \leavevmode
no arguments

\end{description}
\index{sh64\_put\_wired\_dtlb\_entry (C function)}

\begin{fulllineitems}
\phantomsection\label{sh/index:c.sh64_put_wired_dtlb_entry}\pysiglinewithargsret{int \bfcode{sh64\_put\_wired\_dtlb\_entry}}{unsigned long long\emph{ entry}}{}
Free a wired (locked-in) entry in the DTLB.

\end{fulllineitems}


\textbf{Parameters}
\begin{description}
\item[{\code{unsigned long long entry}}] \leavevmode
Address of TLB slot.

\end{description}

\textbf{Description}

Works like a stack, last one to allocate must be first one to free.
\index{sh64\_setup\_tlb\_slot (C function)}

\begin{fulllineitems}
\phantomsection\label{sh/index:c.sh64_setup_tlb_slot}\pysiglinewithargsret{void \bfcode{sh64\_setup\_tlb\_slot}}{unsigned long long\emph{ config\_addr}, unsigned long\emph{ eaddr}, unsigned long\emph{ asid}, unsigned long\emph{ paddr}}{}
Load up a translation in a wired slot.

\end{fulllineitems}


\textbf{Parameters}
\begin{description}
\item[{\code{unsigned long long config\_addr}}] \leavevmode
Address of TLB slot.

\item[{\code{unsigned long eaddr}}] \leavevmode
Virtual address.

\item[{\code{unsigned long asid}}] \leavevmode
Address Space Identifier.

\item[{\code{unsigned long paddr}}] \leavevmode
Physical address.

\end{description}

\textbf{Description}

Load up a virtual\textless{}-\textgreater{}physical translation for \textbf{eaddr**\textless{}-\textgreater{}**paddr} in the
pre-allocated TLB slot \textbf{config\_addr} (see sh64\_get\_wired\_dtlb\_entry).
\index{sh64\_teardown\_tlb\_slot (C function)}

\begin{fulllineitems}
\phantomsection\label{sh/index:c.sh64_teardown_tlb_slot}\pysiglinewithargsret{void \bfcode{sh64\_teardown\_tlb\_slot}}{unsigned long long\emph{ config\_addr}}{}
Teardown a translation.

\end{fulllineitems}


\textbf{Parameters}
\begin{description}
\item[{\code{unsigned long long config\_addr}}] \leavevmode
Address of TLB slot.

\end{description}

\textbf{Description}

Teardown any existing mapping in the TLB slot \textbf{config\_addr}.
\index{for\_each\_dtlb\_entry (C function)}

\begin{fulllineitems}
\phantomsection\label{sh/index:c.for_each_dtlb_entry}\pysiglinewithargsret{\bfcode{for\_each\_dtlb\_entry}}{\emph{tlb}}{}
Iterate over free (non-wired) DTLB entries

\end{fulllineitems}


\textbf{Parameters}
\begin{description}
\item[{\code{tlb}}] \leavevmode
TLB entry

\end{description}
\index{for\_each\_itlb\_entry (C function)}

\begin{fulllineitems}
\phantomsection\label{sh/index:c.for_each_itlb_entry}\pysiglinewithargsret{\bfcode{for\_each\_itlb\_entry}}{\emph{tlb}}{}
Iterate over free (non-wired) ITLB entries

\end{fulllineitems}


\textbf{Parameters}
\begin{description}
\item[{\code{tlb}}] \leavevmode
TLB entry

\end{description}
\index{\_\_flush\_tlb\_slot (C function)}

\begin{fulllineitems}
\phantomsection\label{sh/index:c.__flush_tlb_slot}\pysiglinewithargsret{void \bfcode{\_\_flush\_tlb\_slot}}{unsigned long long\emph{ slot}}{}
Flushes TLB slot \textbf{slot}.

\end{fulllineitems}


\textbf{Parameters}
\begin{description}
\item[{\code{unsigned long long slot}}] \leavevmode
Address of TLB slot.

\end{description}


\chapter{Machine Specific Interfaces}
\label{sh/index:machine-specific-interfaces}

\section{mach-dreamcast}
\label{sh/index:mach-dreamcast}\index{aica\_rtc\_gettimeofday (C function)}

\begin{fulllineitems}
\phantomsection\label{sh/index:c.aica_rtc_gettimeofday}\pysiglinewithargsret{void \bfcode{aica\_rtc\_gettimeofday}}{struct timespec *\emph{ ts}}{}
Get the time from the AICA RTC

\end{fulllineitems}


\textbf{Parameters}
\begin{description}
\item[{\code{struct timespec * ts}}] \leavevmode
pointer to resulting timespec

\end{description}

\textbf{Description}

Grabs the current RTC seconds counter and adjusts it to the Unix Epoch.
\index{aica\_rtc\_settimeofday (C function)}

\begin{fulllineitems}
\phantomsection\label{sh/index:c.aica_rtc_settimeofday}\pysiglinewithargsret{int \bfcode{aica\_rtc\_settimeofday}}{const time\_t\emph{ secs}}{}
Set the AICA RTC to the current time

\end{fulllineitems}


\textbf{Parameters}
\begin{description}
\item[{\code{const time\_t secs}}] \leavevmode
contains the time\_t to set

\end{description}

\textbf{Description}

Adjusts the given \textbf{tv} to the AICA Epoch and sets the RTC seconds counter.


\section{mach-x3proto}
\label{sh/index:mach-x3proto}\index{ilsel\_enable (C function)}

\begin{fulllineitems}
\phantomsection\label{sh/index:c.ilsel_enable}\pysiglinewithargsret{int \bfcode{ilsel\_enable}}{ilsel\_source\_t\emph{ set}}{}
Enable an ILSEL set.

\end{fulllineitems}


\textbf{Parameters}
\begin{description}
\item[{\code{ilsel\_source\_t set}}] \leavevmode
ILSEL source (see ilsel\_source\_t enum in include/asm-sh/ilsel.h).

\end{description}

\textbf{Description}

Enables a given non-aliased ILSEL source (\textless{}= ILSEL\_KEY) at the highest
available interrupt level. Callers should take care to order callsites
noting descending interrupt levels. Aliasing FPGA and external board
IRQs need to use {\hyperref[sh/index:c.ilsel_enable_fixed]{\emph{\code{ilsel\_enable\_fixed()}}}}.

The return value is an IRQ number that can later be taken down with
{\hyperref[sh/index:c.ilsel_disable]{\emph{\code{ilsel\_disable()}}}}.
\index{ilsel\_enable\_fixed (C function)}

\begin{fulllineitems}
\phantomsection\label{sh/index:c.ilsel_enable_fixed}\pysiglinewithargsret{int \bfcode{ilsel\_enable\_fixed}}{ilsel\_source\_t\emph{ set}, unsigned int\emph{ level}}{}
Enable an ILSEL set at a fixed interrupt level

\end{fulllineitems}


\textbf{Parameters}
\begin{description}
\item[{\code{ilsel\_source\_t set}}] \leavevmode
ILSEL source (see ilsel\_source\_t enum in include/asm-sh/ilsel.h).

\item[{\code{unsigned int level}}] \leavevmode
Interrupt level (1 - 15)

\end{description}

\textbf{Description}

Enables a given ILSEL source at a fixed interrupt level. Necessary
both for level reservation as well as for aliased sources that only
exist on special ILSEL\#s.

Returns an IRQ number (as {\hyperref[sh/index:c.ilsel_enable]{\emph{\code{ilsel\_enable()}}}}).
\index{ilsel\_disable (C function)}

\begin{fulllineitems}
\phantomsection\label{sh/index:c.ilsel_disable}\pysiglinewithargsret{void \bfcode{ilsel\_disable}}{unsigned int\emph{ irq}}{}
Disable an ILSEL set

\end{fulllineitems}


\textbf{Parameters}
\begin{description}
\item[{\code{unsigned int irq}}] \leavevmode
Bit position for ILSEL set value (retval from enable routines)

\end{description}

\textbf{Description}

Disable a previously enabled ILSEL set.


\chapter{Busses}
\label{sh/index:busses}

\section{SuperHyway}
\label{sh/index:superhyway}\index{superhyway\_add\_device (C function)}

\begin{fulllineitems}
\phantomsection\label{sh/index:c.superhyway_add_device}\pysiglinewithargsret{int \bfcode{superhyway\_add\_device}}{unsigned long\emph{ base}, struct superhyway\_device *\emph{ sdev}, struct superhyway\_bus *\emph{ bus}}{}
Add a SuperHyway module

\end{fulllineitems}


\textbf{Parameters}
\begin{description}
\item[{\code{unsigned long base}}] \leavevmode
Physical address where module is mapped.

\item[{\code{struct superhyway\_device * sdev}}] \leavevmode
SuperHyway device to add, or NULL to allocate a new one.

\item[{\code{struct superhyway\_bus * bus}}] \leavevmode
Bus where SuperHyway module resides.

\end{description}

\textbf{Description}

This is responsible for adding a new SuperHyway module. This sets up a new
struct superhyway\_device for the module being added if \textbf{sdev} == NULL.

Devices are initially added in the order that they are scanned (from the
top-down of the memory map), and are assigned an ID based on the order that
they are added. Any manual addition of a module will thus get the ID after
the devices already discovered regardless of where it resides in memory.

Further work can and should be done in \code{superhyway\_scan\_bus()}, to be sure
that any new modules are properly discovered and subsequently registered.
\index{superhyway\_register\_driver (C function)}

\begin{fulllineitems}
\phantomsection\label{sh/index:c.superhyway_register_driver}\pysiglinewithargsret{int \bfcode{superhyway\_register\_driver}}{struct superhyway\_driver *\emph{ drv}}{}
Register a new SuperHyway driver

\end{fulllineitems}


\textbf{Parameters}
\begin{description}
\item[{\code{struct superhyway\_driver * drv}}] \leavevmode
SuperHyway driver to register.

\end{description}

\textbf{Description}

This registers the passed in \textbf{drv}. Any devices matching the id table will
automatically be populated and handed off to the driver's specified probe
routine.
\index{superhyway\_unregister\_driver (C function)}

\begin{fulllineitems}
\phantomsection\label{sh/index:c.superhyway_unregister_driver}\pysiglinewithargsret{void \bfcode{superhyway\_unregister\_driver}}{struct superhyway\_driver *\emph{ drv}}{}
Unregister a SuperHyway driver

\end{fulllineitems}


\textbf{Parameters}
\begin{description}
\item[{\code{struct superhyway\_driver * drv}}] \leavevmode
SuperHyway driver to unregister.

\end{description}

\textbf{Description}

This cleans up after {\hyperref[sh/index:c.superhyway_register_driver]{\emph{\code{superhyway\_register\_driver()}}}}, and should be invoked in
the exit path of any module drivers.


\section{Maple}
\label{sh/index:maple}\index{maple\_driver\_register (C function)}

\begin{fulllineitems}
\phantomsection\label{sh/index:c.maple_driver_register}\pysiglinewithargsret{int \bfcode{maple\_driver\_register}}{struct maple\_driver *\emph{ drv}}{}
register a maple driver

\end{fulllineitems}


\textbf{Parameters}
\begin{description}
\item[{\code{struct maple\_driver * drv}}] \leavevmode
maple driver to be registered.

\end{description}

\textbf{Description}

Registers the passed in \textbf{drv}, while updating the bus type.
Devices with matching function IDs will be automatically probed.
\index{maple\_driver\_unregister (C function)}

\begin{fulllineitems}
\phantomsection\label{sh/index:c.maple_driver_unregister}\pysiglinewithargsret{void \bfcode{maple\_driver\_unregister}}{struct maple\_driver *\emph{ drv}}{}
unregister a maple driver.

\end{fulllineitems}


\textbf{Parameters}
\begin{description}
\item[{\code{struct maple\_driver * drv}}] \leavevmode
maple driver to unregister.

\end{description}

\textbf{Description}

Cleans up after {\hyperref[sh/index:c.maple_driver_register]{\emph{\code{maple\_driver\_register()}}}}. To be invoked in the exit
path of any module drivers.
\index{maple\_getcond\_callback (C function)}

\begin{fulllineitems}
\phantomsection\label{sh/index:c.maple_getcond_callback}\pysiglinewithargsret{void \bfcode{maple\_getcond\_callback}}{struct maple\_device *\emph{ dev}, void (*callback) (struct mapleq\emph{ *mq}, unsigned long\emph{ interval}, unsigned long\emph{ function}}{}
setup handling MAPLE\_COMMAND\_GETCOND

\end{fulllineitems}


\textbf{Parameters}
\begin{description}
\item[{\code{struct maple\_device * dev}}] \leavevmode
device responding

\item[{\code{void (*) (struct mapleq *mq) callback}}] \leavevmode
handler callback

\item[{\code{unsigned long interval}}] \leavevmode
interval in jiffies between callbacks

\item[{\code{unsigned long function}}] \leavevmode
the function code for the device

\end{description}
\index{maple\_add\_packet (C function)}

\begin{fulllineitems}
\phantomsection\label{sh/index:c.maple_add_packet}\pysiglinewithargsret{int \bfcode{maple\_add\_packet}}{struct maple\_device *\emph{ mdev}, u32\emph{ function}, u32\emph{ command}, size\_t\emph{ length}, void *\emph{ data}}{}
add a single instruction to the maple bus queue

\end{fulllineitems}


\textbf{Parameters}
\begin{description}
\item[{\code{struct maple\_device * mdev}}] \leavevmode
maple device

\item[{\code{u32 function}}] \leavevmode
function on device being queried

\item[{\code{u32 command}}] \leavevmode
maple command to add

\item[{\code{size\_t length}}] \leavevmode
length of command string (in 32 bit words)

\item[{\code{void * data}}] \leavevmode
remainder of command string

\end{description}



\renewcommand{\indexname}{Index}
\printindex
\end{document}
