% Generated by Sphinx.
\def\sphinxdocclass{report}
\documentclass[a4paper,8pt,english]{sphinxmanual}


\usepackage{cmap}
\usepackage[T1]{fontenc}
\usepackage{amsfonts}
\usepackage{babel}
\usepackage{times}
\usepackage[Bjarne]{fncychap}
\usepackage{longtable}
\usepackage{sphinx}
\usepackage{multirow}
\usepackage{eqparbox}


\addto\captionsenglish{\renewcommand{\figurename}{Fig. }}
\addto\captionsenglish{\renewcommand{\tablename}{Table }}
\SetupFloatingEnvironment{literal-block}{name=Listing }


	% Use some font with UTF-8 support with XeLaTeX
        \usepackage{fontspec}
        \setsansfont{DejaVu Serif}
        \setromanfont{DejaVu Sans}
        \setmonofont{DejaVu Sans Mono}

     \usepackage[margin=0.5in, top=1in, bottom=1in]{geometry}
        \usepackage{ifthen}

        % Put notes in color and let them be inside a table
	\definecolor{NoteColor}{RGB}{204,255,255}
	\definecolor{WarningColor}{RGB}{255,204,204}
	\definecolor{AttentionColor}{RGB}{255,255,204}
	\definecolor{ImportantColor}{RGB}{192,255,204}
	\definecolor{OtherColor}{RGB}{204,204,204}
        \newlength{\mynoticelength}
        \makeatletter\newenvironment{coloredbox}[1]{%
	   \setlength{\fboxrule}{1pt}
	   \setlength{\fboxsep}{7pt}
	   \setlength{\mynoticelength}{\linewidth}
	   \addtolength{\mynoticelength}{-2\fboxsep}
	   \addtolength{\mynoticelength}{-2\fboxrule}
           \begin{lrbox}{\@tempboxa}\begin{minipage}{\mynoticelength}}{\end{minipage}\end{lrbox}%
	   \ifthenelse%
	      {\equal{\py@noticetype}{note}}%
	      {\colorbox{NoteColor}{\usebox{\@tempboxa}}}%
	      {%
	         \ifthenelse%
	         {\equal{\py@noticetype}{warning}}%
	         {\colorbox{WarningColor}{\usebox{\@tempboxa}}}%
		 {%
	            \ifthenelse%
	            {\equal{\py@noticetype}{attention}}%
	            {\colorbox{AttentionColor}{\usebox{\@tempboxa}}}%
		    {%
	               \ifthenelse%
	               {\equal{\py@noticetype}{important}}%
	               {\colorbox{ImportantColor}{\usebox{\@tempboxa}}}%
	               {\colorbox{OtherColor}{\usebox{\@tempboxa}}}%
		    }%
		 }%
	      }%
        }\makeatother

        \makeatletter
        \renewenvironment{notice}[2]{%
          \def\py@noticetype{#1}
          \begin{coloredbox}{#1}
          \bf\it
          \par\strong{#2}
          \csname py@noticestart@#1\endcsname
        }
	{
          \csname py@noticeend@\py@noticetype\endcsname
          \end{coloredbox}
        }
	\makeatother

     

\title{Linux Kernel Documentation Guide}
\date{March 08, 2018}
\release{4.16.0-rc4+}
\author{The kernel development community}
\newcommand{\sphinxlogo}{}
\renewcommand{\releasename}{Release}
\setcounter{tocdepth}{0}
\makeindex

\makeatletter
\def\PYG@reset{\let\PYG@it=\relax \let\PYG@bf=\relax%
    \let\PYG@ul=\relax \let\PYG@tc=\relax%
    \let\PYG@bc=\relax \let\PYG@ff=\relax}
\def\PYG@tok#1{\csname PYG@tok@#1\endcsname}
\def\PYG@toks#1+{\ifx\relax#1\empty\else%
    \PYG@tok{#1}\expandafter\PYG@toks\fi}
\def\PYG@do#1{\PYG@bc{\PYG@tc{\PYG@ul{%
    \PYG@it{\PYG@bf{\PYG@ff{#1}}}}}}}
\def\PYG#1#2{\PYG@reset\PYG@toks#1+\relax+\PYG@do{#2}}

\expandafter\def\csname PYG@tok@gd\endcsname{\def\PYG@tc##1{\textcolor[rgb]{0.63,0.00,0.00}{##1}}}
\expandafter\def\csname PYG@tok@gu\endcsname{\let\PYG@bf=\textbf\def\PYG@tc##1{\textcolor[rgb]{0.50,0.00,0.50}{##1}}}
\expandafter\def\csname PYG@tok@gt\endcsname{\def\PYG@tc##1{\textcolor[rgb]{0.00,0.27,0.87}{##1}}}
\expandafter\def\csname PYG@tok@gs\endcsname{\let\PYG@bf=\textbf}
\expandafter\def\csname PYG@tok@gr\endcsname{\def\PYG@tc##1{\textcolor[rgb]{1.00,0.00,0.00}{##1}}}
\expandafter\def\csname PYG@tok@cm\endcsname{\let\PYG@it=\textit\def\PYG@tc##1{\textcolor[rgb]{0.25,0.50,0.56}{##1}}}
\expandafter\def\csname PYG@tok@vg\endcsname{\def\PYG@tc##1{\textcolor[rgb]{0.73,0.38,0.84}{##1}}}
\expandafter\def\csname PYG@tok@vi\endcsname{\def\PYG@tc##1{\textcolor[rgb]{0.73,0.38,0.84}{##1}}}
\expandafter\def\csname PYG@tok@mh\endcsname{\def\PYG@tc##1{\textcolor[rgb]{0.13,0.50,0.31}{##1}}}
\expandafter\def\csname PYG@tok@cs\endcsname{\def\PYG@tc##1{\textcolor[rgb]{0.25,0.50,0.56}{##1}}\def\PYG@bc##1{\setlength{\fboxsep}{0pt}\colorbox[rgb]{1.00,0.94,0.94}{\strut ##1}}}
\expandafter\def\csname PYG@tok@ge\endcsname{\let\PYG@it=\textit}
\expandafter\def\csname PYG@tok@vc\endcsname{\def\PYG@tc##1{\textcolor[rgb]{0.73,0.38,0.84}{##1}}}
\expandafter\def\csname PYG@tok@il\endcsname{\def\PYG@tc##1{\textcolor[rgb]{0.13,0.50,0.31}{##1}}}
\expandafter\def\csname PYG@tok@go\endcsname{\def\PYG@tc##1{\textcolor[rgb]{0.20,0.20,0.20}{##1}}}
\expandafter\def\csname PYG@tok@cp\endcsname{\def\PYG@tc##1{\textcolor[rgb]{0.00,0.44,0.13}{##1}}}
\expandafter\def\csname PYG@tok@gi\endcsname{\def\PYG@tc##1{\textcolor[rgb]{0.00,0.63,0.00}{##1}}}
\expandafter\def\csname PYG@tok@gh\endcsname{\let\PYG@bf=\textbf\def\PYG@tc##1{\textcolor[rgb]{0.00,0.00,0.50}{##1}}}
\expandafter\def\csname PYG@tok@ni\endcsname{\let\PYG@bf=\textbf\def\PYG@tc##1{\textcolor[rgb]{0.84,0.33,0.22}{##1}}}
\expandafter\def\csname PYG@tok@nl\endcsname{\let\PYG@bf=\textbf\def\PYG@tc##1{\textcolor[rgb]{0.00,0.13,0.44}{##1}}}
\expandafter\def\csname PYG@tok@nn\endcsname{\let\PYG@bf=\textbf\def\PYG@tc##1{\textcolor[rgb]{0.05,0.52,0.71}{##1}}}
\expandafter\def\csname PYG@tok@no\endcsname{\def\PYG@tc##1{\textcolor[rgb]{0.38,0.68,0.84}{##1}}}
\expandafter\def\csname PYG@tok@na\endcsname{\def\PYG@tc##1{\textcolor[rgb]{0.25,0.44,0.63}{##1}}}
\expandafter\def\csname PYG@tok@nb\endcsname{\def\PYG@tc##1{\textcolor[rgb]{0.00,0.44,0.13}{##1}}}
\expandafter\def\csname PYG@tok@nc\endcsname{\let\PYG@bf=\textbf\def\PYG@tc##1{\textcolor[rgb]{0.05,0.52,0.71}{##1}}}
\expandafter\def\csname PYG@tok@nd\endcsname{\let\PYG@bf=\textbf\def\PYG@tc##1{\textcolor[rgb]{0.33,0.33,0.33}{##1}}}
\expandafter\def\csname PYG@tok@ne\endcsname{\def\PYG@tc##1{\textcolor[rgb]{0.00,0.44,0.13}{##1}}}
\expandafter\def\csname PYG@tok@nf\endcsname{\def\PYG@tc##1{\textcolor[rgb]{0.02,0.16,0.49}{##1}}}
\expandafter\def\csname PYG@tok@si\endcsname{\let\PYG@it=\textit\def\PYG@tc##1{\textcolor[rgb]{0.44,0.63,0.82}{##1}}}
\expandafter\def\csname PYG@tok@s2\endcsname{\def\PYG@tc##1{\textcolor[rgb]{0.25,0.44,0.63}{##1}}}
\expandafter\def\csname PYG@tok@nt\endcsname{\let\PYG@bf=\textbf\def\PYG@tc##1{\textcolor[rgb]{0.02,0.16,0.45}{##1}}}
\expandafter\def\csname PYG@tok@nv\endcsname{\def\PYG@tc##1{\textcolor[rgb]{0.73,0.38,0.84}{##1}}}
\expandafter\def\csname PYG@tok@s1\endcsname{\def\PYG@tc##1{\textcolor[rgb]{0.25,0.44,0.63}{##1}}}
\expandafter\def\csname PYG@tok@ch\endcsname{\let\PYG@it=\textit\def\PYG@tc##1{\textcolor[rgb]{0.25,0.50,0.56}{##1}}}
\expandafter\def\csname PYG@tok@m\endcsname{\def\PYG@tc##1{\textcolor[rgb]{0.13,0.50,0.31}{##1}}}
\expandafter\def\csname PYG@tok@gp\endcsname{\let\PYG@bf=\textbf\def\PYG@tc##1{\textcolor[rgb]{0.78,0.36,0.04}{##1}}}
\expandafter\def\csname PYG@tok@sh\endcsname{\def\PYG@tc##1{\textcolor[rgb]{0.25,0.44,0.63}{##1}}}
\expandafter\def\csname PYG@tok@ow\endcsname{\let\PYG@bf=\textbf\def\PYG@tc##1{\textcolor[rgb]{0.00,0.44,0.13}{##1}}}
\expandafter\def\csname PYG@tok@sx\endcsname{\def\PYG@tc##1{\textcolor[rgb]{0.78,0.36,0.04}{##1}}}
\expandafter\def\csname PYG@tok@bp\endcsname{\def\PYG@tc##1{\textcolor[rgb]{0.00,0.44,0.13}{##1}}}
\expandafter\def\csname PYG@tok@c1\endcsname{\let\PYG@it=\textit\def\PYG@tc##1{\textcolor[rgb]{0.25,0.50,0.56}{##1}}}
\expandafter\def\csname PYG@tok@o\endcsname{\def\PYG@tc##1{\textcolor[rgb]{0.40,0.40,0.40}{##1}}}
\expandafter\def\csname PYG@tok@kc\endcsname{\let\PYG@bf=\textbf\def\PYG@tc##1{\textcolor[rgb]{0.00,0.44,0.13}{##1}}}
\expandafter\def\csname PYG@tok@c\endcsname{\let\PYG@it=\textit\def\PYG@tc##1{\textcolor[rgb]{0.25,0.50,0.56}{##1}}}
\expandafter\def\csname PYG@tok@mf\endcsname{\def\PYG@tc##1{\textcolor[rgb]{0.13,0.50,0.31}{##1}}}
\expandafter\def\csname PYG@tok@err\endcsname{\def\PYG@bc##1{\setlength{\fboxsep}{0pt}\fcolorbox[rgb]{1.00,0.00,0.00}{1,1,1}{\strut ##1}}}
\expandafter\def\csname PYG@tok@mb\endcsname{\def\PYG@tc##1{\textcolor[rgb]{0.13,0.50,0.31}{##1}}}
\expandafter\def\csname PYG@tok@ss\endcsname{\def\PYG@tc##1{\textcolor[rgb]{0.32,0.47,0.09}{##1}}}
\expandafter\def\csname PYG@tok@sr\endcsname{\def\PYG@tc##1{\textcolor[rgb]{0.14,0.33,0.53}{##1}}}
\expandafter\def\csname PYG@tok@mo\endcsname{\def\PYG@tc##1{\textcolor[rgb]{0.13,0.50,0.31}{##1}}}
\expandafter\def\csname PYG@tok@kd\endcsname{\let\PYG@bf=\textbf\def\PYG@tc##1{\textcolor[rgb]{0.00,0.44,0.13}{##1}}}
\expandafter\def\csname PYG@tok@mi\endcsname{\def\PYG@tc##1{\textcolor[rgb]{0.13,0.50,0.31}{##1}}}
\expandafter\def\csname PYG@tok@kn\endcsname{\let\PYG@bf=\textbf\def\PYG@tc##1{\textcolor[rgb]{0.00,0.44,0.13}{##1}}}
\expandafter\def\csname PYG@tok@cpf\endcsname{\let\PYG@it=\textit\def\PYG@tc##1{\textcolor[rgb]{0.25,0.50,0.56}{##1}}}
\expandafter\def\csname PYG@tok@kr\endcsname{\let\PYG@bf=\textbf\def\PYG@tc##1{\textcolor[rgb]{0.00,0.44,0.13}{##1}}}
\expandafter\def\csname PYG@tok@s\endcsname{\def\PYG@tc##1{\textcolor[rgb]{0.25,0.44,0.63}{##1}}}
\expandafter\def\csname PYG@tok@kp\endcsname{\def\PYG@tc##1{\textcolor[rgb]{0.00,0.44,0.13}{##1}}}
\expandafter\def\csname PYG@tok@w\endcsname{\def\PYG@tc##1{\textcolor[rgb]{0.73,0.73,0.73}{##1}}}
\expandafter\def\csname PYG@tok@kt\endcsname{\def\PYG@tc##1{\textcolor[rgb]{0.56,0.13,0.00}{##1}}}
\expandafter\def\csname PYG@tok@sc\endcsname{\def\PYG@tc##1{\textcolor[rgb]{0.25,0.44,0.63}{##1}}}
\expandafter\def\csname PYG@tok@sb\endcsname{\def\PYG@tc##1{\textcolor[rgb]{0.25,0.44,0.63}{##1}}}
\expandafter\def\csname PYG@tok@k\endcsname{\let\PYG@bf=\textbf\def\PYG@tc##1{\textcolor[rgb]{0.00,0.44,0.13}{##1}}}
\expandafter\def\csname PYG@tok@se\endcsname{\let\PYG@bf=\textbf\def\PYG@tc##1{\textcolor[rgb]{0.25,0.44,0.63}{##1}}}
\expandafter\def\csname PYG@tok@sd\endcsname{\let\PYG@it=\textit\def\PYG@tc##1{\textcolor[rgb]{0.25,0.44,0.63}{##1}}}

\def\PYGZbs{\char`\\}
\def\PYGZus{\char`\_}
\def\PYGZob{\char`\{}
\def\PYGZcb{\char`\}}
\def\PYGZca{\char`\^}
\def\PYGZam{\char`\&}
\def\PYGZlt{\char`\<}
\def\PYGZgt{\char`\>}
\def\PYGZsh{\char`\#}
\def\PYGZpc{\char`\%}
\def\PYGZdl{\char`\$}
\def\PYGZhy{\char`\-}
\def\PYGZsq{\char`\'}
\def\PYGZdq{\char`\"}
\def\PYGZti{\char`\~}
% for compatibility with earlier versions
\def\PYGZat{@}
\def\PYGZlb{[}
\def\PYGZrb{]}
\makeatother

\renewcommand\PYGZsq{\textquotesingle}

\begin{document}

\maketitle
\tableofcontents
\phantomsection\label{doc-guide/index::doc}



\chapter{Introduction}
\label{doc-guide/sphinx:how-to-write-kernel-documentation}\label{doc-guide/sphinx::doc}\label{doc-guide/sphinx:doc-guide}\label{doc-guide/sphinx:introduction}
The Linux kernel uses \href{http://www.sphinx-doc.org/}{Sphinx} to generate pretty documentation from
\href{http://docutils.sourceforge.net/rst.html}{reStructuredText} files under \code{Documentation}. To build the documentation in
HTML or PDF formats, use \code{make htmldocs} or \code{make pdfdocs}. The generated
documentation is placed in \code{Documentation/output}.

The reStructuredText files may contain directives to include structured
documentation comments, or kernel-doc comments, from source files. Usually these
are used to describe the functions and types and design of the code. The
kernel-doc comments have some special structure and formatting, but beyond that
they are also treated as reStructuredText.

Finally, there are thousands of plain text documentation files scattered around
\code{Documentation}. Some of these will likely be converted to reStructuredText
over time, but the bulk of them will remain in plain text.


\chapter{Sphinx Install}
\label{doc-guide/sphinx:sphinx-install}\label{doc-guide/sphinx:id1}
The ReST markups currently used by the Documentation/ files are meant to be
built with \code{Sphinx} version 1.3 or upper. If you're desiring to build
PDF outputs, it is recommended to use version 1.4.6 or upper.

There's a script that checks for the Spinx requirements. Please see
{\hyperref[doc\string-guide/sphinx:sphinx\string-pre\string-install]{\emph{Checking for Sphinx dependencies}}} for further details.

Most distributions are shipped with Sphinx, but its toolchain is fragile,
and it is not uncommon that upgrading it or some other Python packages
on your machine would cause the documentation build to break.

A way to get rid of that is to use a different version than the one shipped
on your distributions. In order to do that, it is recommended to install
Sphinx inside a virtual environment, using \code{virtualenv-3}
or \code{virtualenv}, depending on how your distribution packaged Python 3.

\begin{notice}{note}{Note:}\begin{enumerate}
\item {} 
Sphinx versions below 1.5 don't work properly with Python's
docutils version 0.13.1 or upper. So, if you're willing to use
those versions, you should run \code{pip install 'docutils==0.12'}.

\item {} 
It is recommended to use the RTD theme for html output. Depending
on the Sphinx version, it should be installed  in separate,
with \code{pip install sphinx\_rtd\_theme}.

\item {} 
Some ReST pages contain math expressions. Due to the way Sphinx work,
those expressions are written using LaTeX notation. It needs texlive
installed with amdfonts and amsmath in order to evaluate them.

\end{enumerate}
\end{notice}

In summary, if you want to install Sphinx version 1.4.9, you should do:

\begin{Verbatim}[commandchars=\\\{\}]
\PYGZdl{} virtualenv sphinx\PYGZus{}1.4
\PYGZdl{} . sphinx\PYGZus{}1.4/bin/activate
(sphinx\PYGZus{}1.4) \PYGZdl{} pip install \PYGZhy{}r Documentation/sphinx/requirements.txt
\end{Verbatim}

After running \code{. sphinx\_1.4/bin/activate}, the prompt will change,
in order to indicate that you're using the new environment. If you
open a new shell, you need to rerun this command to enter again at
the virtual environment before building the documentation.


\section{Image output}
\label{doc-guide/sphinx:image-output}
The kernel documentation build system contains an extension that
handles images on both GraphViz and SVG formats (see
{\hyperref[doc\string-guide/sphinx:sphinx\string-kfigure]{\emph{Figures \& Images}}}).

For it to work, you need to install both GraphViz and ImageMagick
packages. If those packages are not installed, the build system will
still build the documentation, but won't include any images at the
output.


\section{PDF and LaTeX builds}
\label{doc-guide/sphinx:pdf-and-latex-builds}
Such builds are currently supported only with Sphinx versions 1.4 and upper.

For PDF and LaTeX output, you'll also need \code{XeLaTeX} version 3.14159265.

Depending on the distribution, you may also need to install a series of
\code{texlive} packages that provide the minimal set of functionalities
required for \code{XeLaTeX} to work.


\section{Checking for Sphinx dependencies}
\label{doc-guide/sphinx:checking-for-sphinx-dependencies}\label{doc-guide/sphinx:sphinx-pre-install}
There's a script that automatically check for Sphinx dependencies. If it can
recognize your distribution, it will also give a hint about the install
command line options for your distro:

\begin{Verbatim}[commandchars=\\\{\}]
\PYGZdl{} ./scripts/sphinx\PYGZhy{}pre\PYGZhy{}install
Checking if the needed tools for Fedora release 26 (Twenty Six) are available
Warning: better to also install \PYGZdq{}texlive\PYGZhy{}luatex85\PYGZdq{}.
You should run:

        sudo dnf install \PYGZhy{}y texlive\PYGZhy{}luatex85
        /usr/bin/virtualenv sphinx\PYGZus{}1.4
        . sphinx\PYGZus{}1.4/bin/activate
        pip install \PYGZhy{}r Documentation/sphinx/requirements.txt

Can\PYGZsq{}t build as 1 mandatory dependency is missing at ./scripts/sphinx\PYGZhy{}pre\PYGZhy{}install line 468.
\end{Verbatim}

By default, it checks all the requirements for both html and PDF, including
the requirements for images, math expressions and LaTeX build, and assumes
that a virtual Python environment will be used. The ones needed for html
builds are assumed to be mandatory; the others to be optional.

It supports two optional parameters:
\begin{description}
\item[{\code{-{-}no-pdf}}] \leavevmode
Disable checks for PDF;

\item[{\code{-{-}no-virtualenv}}] \leavevmode
Use OS packaging for Sphinx instead of Python virtual environment.

\end{description}


\chapter{Sphinx Build}
\label{doc-guide/sphinx:sphinx-build}
The usual way to generate the documentation is to run \code{make htmldocs} or
\code{make pdfdocs}. There are also other formats available, see the documentation
section of \code{make help}. The generated documentation is placed in
format-specific subdirectories under \code{Documentation/output}.

To generate documentation, Sphinx (\code{sphinx-build}) must obviously be
installed. For prettier HTML output, the Read the Docs Sphinx theme
(\code{sphinx\_rtd\_theme}) is used if available. For PDF output you'll also need
\code{XeLaTeX} and \code{convert(1)} from ImageMagick (\href{https://www.imagemagick.org}{https://www.imagemagick.org}).
All of these are widely available and packaged in distributions.

To pass extra options to Sphinx, you can use the \code{SPHINXOPTS} make
variable. For example, use \code{make SPHINXOPTS=-v htmldocs} to get more verbose
output.

To remove the generated documentation, run \code{make cleandocs}.


\chapter{Writing Documentation}
\label{doc-guide/sphinx:writing-documentation}
Adding new documentation can be as simple as:
\begin{enumerate}
\item {} 
Add a new \code{.rst} file somewhere under \code{Documentation}.

\item {} 
Refer to it from the Sphinx main \href{http://www.sphinx-doc.org/en/stable/markup/toctree.html}{TOC tree} in \code{Documentation/index.rst}.

\end{enumerate}

This is usually good enough for simple documentation (like the one you're
reading right now), but for larger documents it may be advisable to create a
subdirectory (or use an existing one). For example, the graphics subsystem
documentation is under \code{Documentation/gpu}, split to several \code{.rst} files,
and has a separate \code{index.rst} (with a \code{toctree} of its own) referenced from
the main index.

See the documentation for \href{http://www.sphinx-doc.org/}{Sphinx} and \href{http://docutils.sourceforge.net/rst.html}{reStructuredText} on what you can do
with them. In particular, the Sphinx \href{http://www.sphinx-doc.org/en/stable/rest.html}{reStructuredText Primer} is a good place
to get started with reStructuredText. There are also some \href{http://www.sphinx-doc.org/en/stable/markup/index.html}{Sphinx specific
markup constructs}.


\section{Specific guidelines for the kernel documentation}
\label{doc-guide/sphinx:sphinx-specific-markup-constructs}\label{doc-guide/sphinx:specific-guidelines-for-the-kernel-documentation}
Here are some specific guidelines for the kernel documentation:
\begin{itemize}
\item {} 
Please don't go overboard with reStructuredText markup. Keep it
simple. For the most part the documentation should be plain text with
just enough consistency in formatting that it can be converted to
other formats.

\item {} 
Please keep the formatting changes minimal when converting existing
documentation to reStructuredText.

\item {} 
Also update the content, not just the formatting, when converting
documentation.

\item {} 
Please stick to this order of heading adornments:
\begin{enumerate}
\item {} 
\code{=} with overline for document title:

\begin{Verbatim}[commandchars=\\\{\}]
==============
Document title
==============
\end{Verbatim}

\item {} 
\code{=} for chapters:

\begin{Verbatim}[commandchars=\\\{\}]
Chapters
========
\end{Verbatim}

\item {} 
\code{-} for sections:

\begin{Verbatim}[commandchars=\\\{\}]
Section
\PYGZhy{}\PYGZhy{}\PYGZhy{}\PYGZhy{}\PYGZhy{}\PYGZhy{}\PYGZhy{}
\end{Verbatim}

\item {} 
\code{\textasciitilde{}} for subsections:

\begin{Verbatim}[commandchars=\\\{\}]
Subsection
\PYGZti{}\PYGZti{}\PYGZti{}\PYGZti{}\PYGZti{}\PYGZti{}\PYGZti{}\PYGZti{}\PYGZti{}\PYGZti{}
\end{Verbatim}

\end{enumerate}

Although RST doesn't mandate a specific order (``Rather than imposing a fixed
number and order of section title adornment styles, the order enforced will be
the order as encountered.''), having the higher levels the same overall makes
it easier to follow the documents.

\item {} 
For inserting fixed width text blocks (for code examples, use case
examples, etc.), use \code{::} for anything that doesn't really benefit
from syntax highlighting, especially short snippets. Use
\code{.. code-block:: \textless{}language\textgreater{}} for longer code blocks that benefit
from highlighting.

\end{itemize}


\section{the C domain}
\label{doc-guide/sphinx:the-c-domain}
The \textbf{Sphinx C Domain} (name c) is suited for documentation of C API. E.g. a
function prototype:

\begin{Verbatim}[commandchars=\\\{\}]
\PYG{p}{..} \PYG{o+ow}{c:function}\PYG{p}{::} int ioctl( int fd, int request )
\end{Verbatim}

The C domain of the kernel-doc has some additional features. E.g. you can
\emph{rename} the reference name of a function with a common name like \code{open} or
\code{ioctl}:

\begin{Verbatim}[commandchars=\\\{\}]
\PYG{p}{..} \PYG{o+ow}{c:function}\PYG{p}{::} int ioctl( int fd, int request )
   \PYG{n+nc}{:name:} \PYG{n+nf}{VIDIOC\PYGZus{}LOG\PYGZus{}STATUS}
\end{Verbatim}

The func-name (e.g. ioctl) remains in the output but the ref-name changed from
\code{ioctl} to \code{VIDIOC\_LOG\_STATUS}. The index entry for this function is also
changed to \code{VIDIOC\_LOG\_STATUS} and the function can now referenced by:

\begin{Verbatim}[commandchars=\\\{\}]
\PYG{n+na}{:c:func:}\PYG{n+nv}{{}`VIDIOC\PYGZus{}LOG\PYGZus{}STATUS{}`}
\end{Verbatim}


\section{list tables}
\label{doc-guide/sphinx:list-tables}
We recommend the use of \emph{list table} formats. The \emph{list table} formats are
double-stage lists. Compared to the ASCII-art they might not be as
comfortable for
readers of the text files. Their advantage is that they are easy to
create or modify and that the diff of a modification is much more meaningful,
because it is limited to the modified content.

The \code{flat-table} is a double-stage list similar to the \code{list-table} with
some additional features:
\begin{itemize}
\item {} 
column-span: with the role \code{cspan} a cell can be extended through
additional columns

\item {} 
row-span: with the role \code{rspan} a cell can be extended through
additional rows

\item {} 
auto span rightmost cell of a table row over the missing cells on the right
side of that table-row.  With Option \code{:fill-cells:} this behavior can
changed from \emph{auto span} to \emph{auto fill}, which automatically inserts (empty)
cells instead of spanning the last cell.

\end{itemize}

options:
\begin{itemize}
\item {} 
\code{:header-rows:}   {[}int{]} count of header rows

\item {} 
\code{:stub-columns:}  {[}int{]} count of stub columns

\item {} 
\code{:widths:}        {[}{[}int{]} {[}int{]} ... {]} widths of columns

\item {} 
\code{:fill-cells:}    instead of auto-spanning missing cells, insert missing cells

\end{itemize}

roles:
\begin{itemize}
\item {} 
\code{:cspan:} {[}int{]} additional columns (\emph{morecols})

\item {} 
\code{:rspan:} {[}int{]} additional rows (\emph{morerows})

\end{itemize}

The example below shows how to use this markup.  The first level of the staged
list is the \emph{table-row}. In the \emph{table-row} there is only one markup allowed,
the list of the cells in this \emph{table-row}. Exceptions are \emph{comments} ( \code{..} )
and \emph{targets} (e.g. a ref to \code{:ref:{}`last row \textless{}last row\textgreater{}{}`} / {\hyperref[doc\string-guide/sphinx:last\string-row]{\emph{last row}}}).

\begin{Verbatim}[commandchars=\\\{\}]
\PYG{p}{..} \PYG{o+ow}{flat\PYGZhy{}table}\PYG{p}{::} table title
   \PYG{n+nc}{:widths:} \PYG{n+nf}{2 1 1 3}

   \PYG{l+m}{*} \PYGZhy{} head col 1
     \PYG{l+m}{\PYGZhy{}} head col 2
     \PYG{l+m}{\PYGZhy{}} head col 3
     \PYG{l+m}{\PYGZhy{}} head col 4

   \PYG{l+m}{*} \PYGZhy{} column 1
     \PYG{l+m}{\PYGZhy{}} field 1.1
     \PYG{l+m}{\PYGZhy{}} field 1.2 with autospan

   \PYG{l+m}{*} \PYGZhy{} column 2
     \PYG{l+m}{\PYGZhy{}} field 2.1
     \PYG{l+m}{\PYGZhy{}} \PYG{n+na}{:rspan:}\PYG{n+nv}{{}`1{}`} \PYG{n+na}{:cspan:}\PYG{n+nv}{{}`1{}`} field 2.2 \PYGZhy{} 3.3

   \PYG{l+m}{*} .. \PYGZus{}\PYG{n+nv}{{}`last row{}`}:

     \PYG{l+m}{\PYGZhy{}} column 3
\end{Verbatim}

Rendered as:
\begin{quote}


\begin{threeparttable}
\capstart\caption{table title}\label{doc-guide/sphinx:id2}
\begin{tabulary}{\linewidth}{|L|L|L|L|}
\hline

head col 1
 & 
head col 2
 & 
head col 3
 & 
head col 4
\\
\hline
column 1
 & 
field 1.1
 &  \multicolumn{2}{l|}{
field 1.2 with autospan
}\\
\hline
column 2
 & 
field 2.1
 &  \multicolumn{2}{l|}{ \multirow{2}{*}{
  field 2.2 - 3.3
}}\\
\cline{1-2}\phantomsection\label{doc-guide/sphinx:last-row}
column 3
 &  &  \multicolumn{2}{l|}{}\\
\hline\end{tabulary}

\end{threeparttable}

\end{quote}


\chapter{Figures \& Images}
\label{doc-guide/sphinx:sphinx-kfigure}\label{doc-guide/sphinx:figures-images}
If you want to add an image, you should use the \code{kernel-figure} and
\code{kernel-image} directives. E.g. to insert a figure with a scalable
image format use SVG ({\hyperref[doc\string-guide/sphinx:svg\string-image\string-example]{\emph{SVG image example}}}):

\begin{Verbatim}[commandchars=\\\{\}]
.. kernel\PYGZhy{}figure::  svg\PYGZus{}image.svg
   :alt:    simple SVG image

   SVG image example
\end{Verbatim}
\begin{figure}[htbp]
\centering
\capstart

\includegraphics{{svg_image}.pdf}
\caption{SVG image example}\label{doc-guide/sphinx:svg-image-example}\end{figure}

The kernel figure (and image) directive support \textbf{DOT} formated files, see
\begin{itemize}
\item {} 
DOT: \href{http://graphviz.org/pdf/dotguide.pdf}{http://graphviz.org/pdf/dotguide.pdf}

\item {} 
Graphviz: \href{http://www.graphviz.org/content/dot-language}{http://www.graphviz.org/content/dot-language}

\end{itemize}

A simple example ({\hyperref[doc\string-guide/sphinx:hello\string-dot\string-file]{\emph{DOT's hello world example}}}):

\begin{Verbatim}[commandchars=\\\{\}]
.. kernel\PYGZhy{}figure::  hello.dot
   :alt:    hello world

   DOT\PYGZsq{}s hello world example
\end{Verbatim}
\begin{figure}[htbp]
\centering
\capstart

\includegraphics{{hello}.pdf}
\caption{DOT's hello world example}\label{doc-guide/sphinx:hello-dot-file}\end{figure}

Embed \emph{render} markups (or languages) like Graphviz's \textbf{DOT} is provided by the
\code{kernel-render} directives.:

\begin{Verbatim}[commandchars=\\\{\}]
.. kernel\PYGZhy{}render:: DOT
   :alt: foobar digraph
   :caption: Embedded **DOT** (Graphviz) code

   digraph foo \PYGZob{}
    \PYGZdq{}bar\PYGZdq{} \PYGZhy{}\PYGZgt{} \PYGZdq{}baz\PYGZdq{};
   \PYGZcb{}
\end{Verbatim}

How this will be rendered depends on the installed tools. If Graphviz is
installed, you will see an vector image. If not the raw markup is inserted as
\emph{literal-block} ({\hyperref[doc\string-guide/sphinx:hello\string-dot\string-render]{\emph{Embedded DOT (Graphviz) code}}}).
\begin{figure}[htbp]
\centering
\capstart

\includegraphics{{DOT-1e38538b2ff52f303e5ecd4abe763316501a0a12}.pdf}
\caption{Embedded \textbf{DOT} (Graphviz) code}\label{doc-guide/sphinx:hello-dot-render}\end{figure}

The \emph{render} directive has all the options known from the \emph{figure} directive,
plus option \code{caption}.  If \code{caption} has a value, a \emph{figure} node is
inserted. If not, a \emph{image} node is inserted. A \code{caption} is also needed, if
you want to refer it ({\hyperref[doc\string-guide/sphinx:hello\string-svg\string-render]{\emph{Embedded SVG markup}}}).

Embedded \textbf{SVG}:

\begin{Verbatim}[commandchars=\\\{\}]
.. kernel\PYGZhy{}render:: SVG
   :caption: Embedded **SVG** markup
   :alt: so\PYGZhy{}nw\PYGZhy{}arrow

   \PYGZlt{}?xml version=\PYGZdq{}1.0\PYGZdq{} encoding=\PYGZdq{}UTF\PYGZhy{}8\PYGZdq{}?\PYGZgt{}
   \PYGZlt{}svg xmlns=\PYGZdq{}http://www.w3.org/2000/svg\PYGZdq{} version=\PYGZdq{}1.1\PYGZdq{} ...\PYGZgt{}
      ...
   \PYGZlt{}/svg\PYGZgt{}
\end{Verbatim}
\begin{figure}[htbp]
\centering
\capstart

\includegraphics{{SVG-3bcc8521eb73e7bcc8b0de14afcc31d733ec6b25}.pdf}
\caption{Embedded \textbf{SVG} markup}\label{doc-guide/sphinx:hello-svg-render}\end{figure}


\chapter{Including kernel-doc comments}
\label{doc-guide/kernel-doc::doc}\label{doc-guide/kernel-doc:including-kernel-doc-comments}
The Linux kernel source files may contain structured documentation comments, or
kernel-doc comments to describe the functions and types and design of the
code. The documentation comments may be included to any of the reStructuredText
documents using a dedicated kernel-doc Sphinx directive extension.

The kernel-doc directive is of the format:

\begin{Verbatim}[commandchars=\\\{\}]
.. kernel\PYGZhy{}doc:: source
   :option:
\end{Verbatim}

The \emph{source} is the path to a source file, relative to the kernel source
tree. The following directive options are supported:
\begin{description}
\item[{export: \emph{{[}source-pattern ...{]}}}] \leavevmode
Include documentation for all functions in \emph{source} that have been exported
using \code{EXPORT\_SYMBOL} or \code{EXPORT\_SYMBOL\_GPL} either in \emph{source} or in any
of the files specified by \emph{source-pattern}.

The \emph{source-pattern} is useful when the kernel-doc comments have been placed
in header files, while \code{EXPORT\_SYMBOL} and \code{EXPORT\_SYMBOL\_GPL} are next to
the function definitions.

Examples:

\begin{Verbatim}[commandchars=\\\{\}]
.. kernel\PYGZhy{}doc:: lib/bitmap.c
   :export:

.. kernel\PYGZhy{}doc:: include/net/mac80211.h
   :export: net/mac80211/*.c
\end{Verbatim}

\item[{internal: \emph{{[}source-pattern ...{]}}}] \leavevmode
Include documentation for all functions and types in \emph{source} that have
\textbf{not} been exported using \code{EXPORT\_SYMBOL} or \code{EXPORT\_SYMBOL\_GPL} either
in \emph{source} or in any of the files specified by \emph{source-pattern}.

Example:

\begin{Verbatim}[commandchars=\\\{\}]
.. kernel\PYGZhy{}doc:: drivers/gpu/drm/i915/intel\PYGZus{}audio.c
   :internal:
\end{Verbatim}

\item[{doc: \emph{title}}] \leavevmode
Include documentation for the \code{DOC:} paragraph identified by \emph{title} in
\emph{source}. Spaces are allowed in \emph{title}; do not quote the \emph{title}. The \emph{title}
is only used as an identifier for the paragraph, and is not included in the
output. Please make sure to have an appropriate heading in the enclosing
reStructuredText document.

Example:

\begin{Verbatim}[commandchars=\\\{\}]
.. kernel\PYGZhy{}doc:: drivers/gpu/drm/i915/intel\PYGZus{}audio.c
   :doc: High Definition Audio over HDMI and Display Port
\end{Verbatim}

\item[{functions: \emph{function} \emph{{[}...{]}}}] \leavevmode
Include documentation for each \emph{function} in \emph{source}.

Example:

\begin{Verbatim}[commandchars=\\\{\}]
.. kernel\PYGZhy{}doc:: lib/bitmap.c
   :functions: bitmap\PYGZus{}parselist bitmap\PYGZus{}parselist\PYGZus{}user
\end{Verbatim}

\end{description}

Without options, the kernel-doc directive includes all documentation comments
from the source file.

The kernel-doc extension is included in the kernel source tree, at
\code{Documentation/sphinx/kerneldoc.py}. Internally, it uses the
\code{scripts/kernel-doc} script to extract the documentation comments from the
source.


\chapter{Writing kernel-doc comments}
\label{doc-guide/kernel-doc:kernel-doc}\label{doc-guide/kernel-doc:writing-kernel-doc-comments}
In order to provide embedded, ``C'' friendly, easy to maintain, but consistent and
extractable overview, function and type documentation, the Linux kernel has
adopted a consistent style for documentation comments. The format for this
documentation is called the kernel-doc format, described below. This style
embeds the documentation within the source files, using a few simple conventions
for adding documentation paragraphs and documenting functions and their
parameters, structures and unions and their members, enumerations, and typedefs.

\begin{notice}{note}{Note:}
The kernel-doc format is deceptively similar to gtk-doc or Doxygen,
yet distinctively different, for historical reasons. The kernel source
contains tens of thousands of kernel-doc comments. Please stick to the style
described here.
\end{notice}

The \code{scripts/kernel-doc} script is used by the Sphinx kernel-doc extension in
the documentation build to extract this embedded documentation into the various
HTML, PDF, and other format documents.

In order to provide good documentation of kernel functions and data structures,
please use the following conventions to format your kernel-doc comments in the
Linux kernel source.


\section{How to format kernel-doc comments}
\label{doc-guide/kernel-doc:how-to-format-kernel-doc-comments}
The opening comment mark \code{/**} is reserved for kernel-doc comments. Only
comments so marked will be considered by the \code{kernel-doc} tool. Use it only
for comment blocks that contain kernel-doc formatted comments. The usual \code{*/}
should be used as the closing comment marker. The lines in between should be
prefixed by \code{ * } (space star space).

The function and type kernel-doc comments should be placed just before the
function or type being described. The overview kernel-doc comments may be freely
placed at the top indentation level.

Example kernel-doc function comment:

\begin{Verbatim}[commandchars=\\\{\}]
/**
 * foobar() \PYGZhy{} Brief description of foobar.
 * @argument1: Description of parameter argument1 of foobar.
 * @argument2: Description of parameter argument2 of foobar.
 *
 * Longer description of foobar.
 *
 * Return: Description of return value of foobar.
 */
int foobar(int argument1, char *argument2)
\end{Verbatim}

The format is similar for documentation for structures, enums, paragraphs,
etc. See the sections below for specific details of each type.

The kernel-doc structure is extracted from the comments, and proper \href{http://www.sphinx-doc.org/en/stable/domains.html}{Sphinx C
Domain} function and type descriptions with anchors are generated for them. The
descriptions are filtered for special kernel-doc highlights and
cross-references. See below for details.


\section{Parameters and member arguments}
\label{doc-guide/kernel-doc:sphinx-c-domain}\label{doc-guide/kernel-doc:parameters-and-member-arguments}
The kernel-doc function comments describe each parameter to the function and
function typedefs or each member of struct/union, in order, with the
\code{@argument:} descriptions. For each non-private member argument, one
\code{@argument} definition is needed.

The \code{@argument:} descriptions begin on the very next line following
the opening brief function description line, with no intervening blank
comment lines.

The \code{@argument:} descriptions may span multiple lines.

\begin{notice}{note}{Note:}
If the \code{@argument} description has multiple lines, the continuation
of the description should be starting exactly at the same column as
the previous line, e. g.:

\begin{Verbatim}[commandchars=\\\{\}]
* @argument: some long description
*       that continues on next lines
\end{Verbatim}

or:

\begin{Verbatim}[commandchars=\\\{\}]
* @argument:
*         some long description
*         that continues on next lines
\end{Verbatim}
\end{notice}

If a function or typedef parameter argument is \code{...} (e. g. a variable
number of arguments), its description should be listed in kernel-doc
notation as:

\begin{Verbatim}[commandchars=\\\{\}]
* @...: description
\end{Verbatim}


\subsection{Private members}
\label{doc-guide/kernel-doc:private-members}
Inside a struct or union description, you can use the \code{private:} and
\code{public:} comment tags. Structure fields that are inside a \code{private:}
area are not listed in the generated output documentation.

The \code{private:} and \code{public:} tags must begin immediately following a
\code{/*} comment marker.  They may optionally include comments between the
\code{:} and the ending \code{*/} marker.

Example:

\begin{Verbatim}[commandchars=\\\{\}]
/**
 * struct my\PYGZus{}struct \PYGZhy{} short description
 * @a: first member
 * @b: second member
 * @d: fourth member
 *
 * Longer description
 */
struct my\PYGZus{}struct \PYGZob{}
    int a;
    int b;
/* private: internal use only */
    int c;
/* public: the next one is public */
    int d;
\PYGZcb{};
\end{Verbatim}


\section{Function documentation}
\label{doc-guide/kernel-doc:function-documentation}
The general format of a function and function-like macro kernel-doc comment is:

\begin{Verbatim}[commandchars=\\\{\}]
/**
 * function\PYGZus{}name() \PYGZhy{} Brief description of function.
 * @arg1: Describe the first argument.
 * @arg2: Describe the second argument.
 *        One can provide multiple line descriptions
 *        for arguments.
 *
 * A longer description, with more discussion of the function function\PYGZus{}name()
 * that might be useful to those using or modifying it. Begins with an
 * empty comment line, and may include additional embedded empty
 * comment lines.
 *
 * The longer description may have multiple paragraphs.
 *
 * Return: Describe the return value of foobar.
 *
 * The return value description can also have multiple paragraphs, and should
 * be placed at the end of the comment block.
 */
\end{Verbatim}

The brief description following the function name may span multiple lines, and
ends with an argument description, a blank comment line, or the end of the
comment block.


\subsection{Return values}
\label{doc-guide/kernel-doc:return-values}
The return value, if any, should be described in a dedicated section
named \code{Return}.

\begin{notice}{note}{Note:}\begin{enumerate}
\item {} 
The multi-line descriptive text you provide does \emph{not} recognize
line breaks, so if you try to format some text nicely, as in:

\begin{Verbatim}[commandchars=\\\{\}]
* Return:
* 0 \PYGZhy{} OK
* \PYGZhy{}EINVAL \PYGZhy{} invalid argument
* \PYGZhy{}ENOMEM \PYGZhy{} out of memory
\end{Verbatim}

this will all run together and produce:

\begin{Verbatim}[commandchars=\\\{\}]
Return: 0 \PYGZhy{} OK \PYGZhy{}EINVAL \PYGZhy{} invalid argument \PYGZhy{}ENOMEM \PYGZhy{} out of memory
\end{Verbatim}

So, in order to produce the desired line breaks, you need to use a
ReST list, e. g.:

\begin{Verbatim}[commandchars=\\\{\}]
* Return:
* * 0             \PYGZhy{} OK to runtime suspend the device
* * \PYGZhy{}EBUSY        \PYGZhy{} Device should not be runtime suspended
\end{Verbatim}

\item {} 
If the descriptive text you provide has lines that begin with
some phrase followed by a colon, each of those phrases will be taken
as a new section heading, with probably won't produce the desired
effect.

\end{enumerate}
\end{notice}


\section{Structure, union, and enumeration documentation}
\label{doc-guide/kernel-doc:structure-union-and-enumeration-documentation}
The general format of a struct, union, and enum kernel-doc comment is:

\begin{Verbatim}[commandchars=\\\{\}]
/**
 * struct struct\PYGZus{}name \PYGZhy{} Brief description.
 * @argument: Description of member member\PYGZus{}name.
 *
 * Description of the structure.
 */
\end{Verbatim}

On the above, \code{struct} is used to mean structs. You can also use \code{union}
and \code{enum}  to describe unions and enums. \code{argument} is used
to mean struct and union member names as well as enumerations in an enum.

The brief description following the structure name may span multiple lines, and
ends with a member description, a blank comment line, or the end of the
comment block.

The kernel-doc data structure comments describe each member of the structure,
in order, with the member descriptions.


\subsection{Nested structs/unions}
\label{doc-guide/kernel-doc:nested-structs-unions}
It is possible to document nested structs unions, like:

\begin{Verbatim}[commandchars=\\\{\}]
/**
 * struct nested\PYGZus{}foobar \PYGZhy{} a struct with nested unions and structs
 * @arg1: \PYGZhy{} first argument of anonymous union/anonymous struct
 * @arg2: \PYGZhy{} second argument of anonymous union/anonymous struct
 * @arg3: \PYGZhy{} third argument of anonymous union/anonymous struct
 * @arg4: \PYGZhy{} fourth argument of anonymous union/anonymous struct
 * @bar.st1.arg1 \PYGZhy{} first argument of struct st1 on union bar
 * @bar.st1.arg2 \PYGZhy{} second argument of struct st1 on union bar
 * @bar.st2.arg1 \PYGZhy{} first argument of struct st2 on union bar
 * @bar.st2.arg2 \PYGZhy{} second argument of struct st2 on union bar
struct nested\PYGZus{}foobar \PYGZob{}
  /* Anonymous union/struct*/
  union \PYGZob{}
    struct \PYGZob{}
      int arg1;
      int arg2;
    \PYGZcb{}
    struct \PYGZob{}
      void *arg3;
      int arg4;
    \PYGZcb{}
  \PYGZcb{}
  union \PYGZob{}
    struct \PYGZob{}
      int arg1;
      int arg2;
    \PYGZcb{} st1;
    struct \PYGZob{}
      void *arg1;
      int arg2;
    \PYGZcb{} st2;
  \PYGZcb{} bar;
\PYGZcb{};
\end{Verbatim}

\begin{notice}{note}{Note:}\begin{enumerate}
\item {} 
When documenting nested structs or unions, if the struct/union \code{foo}
is named, the argument \code{bar} inside it should be documented as
\code{@foo.bar:}

\item {} 
When the nested struct/union is anonymous, the argument \code{bar} on it
should be documented as \code{@bar:}

\end{enumerate}
\end{notice}


\section{Typedef documentation}
\label{doc-guide/kernel-doc:typedef-documentation}
The general format of a typedef kernel-doc comment is:

\begin{Verbatim}[commandchars=\\\{\}]
/**
 * typedef type\PYGZus{}name \PYGZhy{} Brief description.
 *
 * Description of the type.
 */
\end{Verbatim}

Typedefs with function prototypes can also be documented:

\begin{Verbatim}[commandchars=\\\{\}]
/**
 * typedef type\PYGZus{}name \PYGZhy{} Brief description.
 * @arg1: description of arg1
 * @arg2: description of arg2
 *
 * Description of the type.
 */
 typedef void (*type\PYGZus{}name)(struct v4l2\PYGZus{}ctrl *arg1, void *arg2);
\end{Verbatim}


\section{Highlights and cross-references}
\label{doc-guide/kernel-doc:highlights-and-cross-references}
The following special patterns are recognized in the kernel-doc comment
descriptive text and converted to proper reStructuredText markup and \href{http://www.sphinx-doc.org/en/stable/domains.html}{Sphinx C
Domain} references.

\begin{notice}{attention}{Attention:}
The below are \textbf{only} recognized within kernel-doc comments,
\textbf{not} within normal reStructuredText documents.
\end{notice}
\begin{description}
\item[{\code{funcname()}}] \leavevmode
Function reference.

\item[{\code{@parameter}}] \leavevmode
Name of a function parameter. (No cross-referencing, just formatting.)

\item[{\code{\%CONST}}] \leavevmode
Name of a constant. (No cross-referencing, just formatting.)

\item[{\code{{}`{}`literal{}`{}`}}] \leavevmode
A literal block that should be handled as-is. The output will use a
\code{monospaced font}.

Useful if you need to use special characters that would otherwise have some
meaning either by kernel-doc script of by reStructuredText.

This is particularly useful if you need to use things like \code{\%ph} inside
a function description.

\item[{\code{\$ENVVAR}}] \leavevmode
Name of an environment variable. (No cross-referencing, just formatting.)

\item[{\code{\&struct name}}] \leavevmode
Structure reference.

\item[{\code{\&enum name}}] \leavevmode
Enum reference.

\item[{\code{\&typedef name}}] \leavevmode
Typedef reference.

\item[{\code{\&struct\_name-\textgreater{}member} or \code{\&struct\_name.member}}] \leavevmode
Structure or union member reference. The cross-reference will be to the struct
or union definition, not the member directly.

\item[{\code{\&name}}] \leavevmode
A generic type reference. Prefer using the full reference described above
instead. This is mostly for legacy comments.

\end{description}


\subsection{Cross-referencing from reStructuredText}
\label{doc-guide/kernel-doc:cross-referencing-from-restructuredtext}
To cross-reference the functions and types defined in the kernel-doc comments
from reStructuredText documents, please use the \href{http://www.sphinx-doc.org/en/stable/domains.html}{Sphinx C Domain}
references. For example:

\begin{Verbatim}[commandchars=\\\{\}]
See function :c:func:{}`foo{}` and struct/union/enum/typedef :c:type:{}`bar{}`.
\end{Verbatim}

While the type reference works with just the type name, without the
struct/union/enum/typedef part in front, you may want to use:

\begin{Verbatim}[commandchars=\\\{\}]
See :c:type:{}`struct foo \PYGZlt{}foo\PYGZgt{}{}`.
See :c:type:{}`union bar \PYGZlt{}bar\PYGZgt{}{}`.
See :c:type:{}`enum baz \PYGZlt{}baz\PYGZgt{}{}`.
See :c:type:{}`typedef meh \PYGZlt{}meh\PYGZgt{}{}`.
\end{Verbatim}

This will produce prettier links, and is in line with how kernel-doc does the
cross-references.

For further details, please refer to the \href{http://www.sphinx-doc.org/en/stable/domains.html}{Sphinx C Domain} documentation.


\subsection{In-line member documentation comments}
\label{doc-guide/kernel-doc:in-line-member-documentation-comments}
The structure members may also be documented in-line within the definition.
There are two styles, single-line comments where both the opening \code{/**} and
closing \code{*/} are on the same line, and multi-line comments where they are each
on a line of their own, like all other kernel-doc comments:

\begin{Verbatim}[commandchars=\\\{\}]
/**
 * struct foo \PYGZhy{} Brief description.
 * @foo: The Foo member.
 */
struct foo \PYGZob{}
      int foo;
      /**
       * @bar: The Bar member.
       */
      int bar;
      /**
       * @baz: The Baz member.
       *
       * Here, the member description may contain several paragraphs.
       */
      int baz;
      /** @foobar: Single line description. */
      int foobar;
\PYGZcb{}
\end{Verbatim}


\section{Overview documentation comments}
\label{doc-guide/kernel-doc:overview-documentation-comments}
To facilitate having source code and comments close together, you can include
kernel-doc documentation blocks that are free-form comments instead of being
kernel-doc for functions, structures, unions, enums, or typedefs. This could be
used for something like a theory of operation for a driver or library code, for
example.

This is done by using a \code{DOC:} section keyword with a section title.

The general format of an overview or high-level documentation comment is:

\begin{Verbatim}[commandchars=\\\{\}]
/**
 * DOC: Theory of Operation
 *
 * The whizbang foobar is a dilly of a gizmo. It can do whatever you
 * want it to do, at any time. It reads your mind. Here\PYGZsq{}s how it works.
 *
 * foo bar splat
 *
 * The only drawback to this gizmo is that is can sometimes damage
 * hardware, software, or its subject(s).
 */
\end{Verbatim}

The title following \code{DOC:} acts as a heading within the source file, but also
as an identifier for extracting the documentation comment. Thus, the title must
be unique within the file.


\section{Recommendations}
\label{doc-guide/kernel-doc:recommendations}
We definitely need kernel-doc formatted documentation for functions that are
exported to loadable modules using \code{EXPORT\_SYMBOL} or \code{EXPORT\_SYMBOL\_GPL}.

We also look to provide kernel-doc formatted documentation for functions
externally visible to other kernel files (not marked ``static'').

We also recommend providing kernel-doc formatted documentation for private (file
``static'') routines, for consistency of kernel source code layout. But this is
lower priority and at the discretion of the MAINTAINER of that kernel source
file.

Data structures visible in kernel include files should also be documented using
kernel-doc formatted comments.


\section{How to use kernel-doc to generate man pages}
\label{doc-guide/kernel-doc:how-to-use-kernel-doc-to-generate-man-pages}
If you just want to use kernel-doc to generate man pages you can do this
from the Kernel git tree:

\begin{Verbatim}[commandchars=\\\{\}]
\PYGZdl{} scripts/kernel\PYGZhy{}doc \PYGZhy{}man \PYGZdl{}(git grep \PYGZhy{}l \PYGZsq{}/\PYGZbs{}*\PYGZbs{}*\PYGZsq{} \textbar{}grep \PYGZhy{}v Documentation/) \textbar{} ./split\PYGZhy{}man.pl /tmp/man
\end{Verbatim}

Using the small \code{split-man.pl} script below:

\begin{Verbatim}[commandchars=\\\{\}]
\PYGZsh{}!/usr/bin/perl

if (\PYGZdl{}\PYGZsh{}ARGV \PYGZlt{} 0) \PYGZob{}
   die \PYGZdq{}where do I put the results?\PYGZbs{}n\PYGZdq{};
\PYGZcb{}

mkdir \PYGZdl{}ARGV[0],0777;
\PYGZdl{}state = 0;
while (\PYGZlt{}STDIN\PYGZgt{}) \PYGZob{}
    if (/\PYGZca{}\PYGZbs{}.TH \PYGZbs{}\PYGZdq{}[\PYGZca{}\PYGZbs{}\PYGZdq{}]*\PYGZbs{}\PYGZdq{} 9 \PYGZbs{}\PYGZdq{}([\PYGZca{}\PYGZbs{}\PYGZdq{}]*)\PYGZbs{}\PYGZdq{}/) \PYGZob{}
      if (\PYGZdl{}state == 1) \PYGZob{} close OUT \PYGZcb{}
      \PYGZdl{}state = 1;
      \PYGZdl{}fn = \PYGZdq{}\PYGZdl{}ARGV[0]/\PYGZdl{}1.9\PYGZdq{};
      print STDERR \PYGZdq{}Creating \PYGZdl{}fn\PYGZbs{}n\PYGZdq{};
      open OUT, \PYGZdq{}\PYGZgt{}\PYGZdl{}fn\PYGZdq{} or die \PYGZdq{}can\PYGZsq{}t open \PYGZdl{}fn: \PYGZdl{}!\PYGZbs{}n\PYGZdq{};
      print OUT \PYGZdl{}\PYGZus{};
    \PYGZcb{} elsif (\PYGZdl{}state != 0) \PYGZob{}
      print OUT \PYGZdl{}\PYGZus{};
    \PYGZcb{}
\PYGZcb{}

close OUT;
\end{Verbatim}


\chapter{Including uAPI header files}
\label{doc-guide/parse-headers:including-uapi-header-files}\label{doc-guide/parse-headers::doc}
Sometimes, it is useful to include header files and C example codes in
order to describe the userspace API and to generate cross-references
between the code and the documentation. Adding cross-references for
userspace API files has an additional vantage: Sphinx will generate warnings
if a symbol is not found at the documentation. That helps to keep the
uAPI documentation in sync with the Kernel changes.
The {\hyperref[doc\string-guide/parse\string-headers:parse\string-headers]{\emph{parse\_headers.pl}}} provide a way to generate such
cross-references. It has to be called via Makefile, while building the
documentation. Please see \code{Documentation/media/Makefile} for an example
about how to use it inside the Kernel tree.


\section{parse\_headers.pl}
\label{doc-guide/parse-headers:parse-headers}\label{doc-guide/parse-headers:parse-headers-pl}

\subsection{NAME}
\label{doc-guide/parse-headers:name}
parse\_headers.pl - parse a C file, in order to identify functions, structs,
enums and defines and create cross-references to a Sphinx book.


\subsection{SYNOPSIS}
\label{doc-guide/parse-headers:synopsis}
\textbf{parse\_headers.pl} {[}\textless{}options\textgreater{}{]} \textless{}C\_FILE\textgreater{} \textless{}OUT\_FILE\textgreater{} {[}\textless{}EXCEPTIONS\_FILE\textgreater{}{]}

Where \textless{}options\textgreater{} can be: --debug, --help or --man.


\subsection{OPTIONS}
\label{doc-guide/parse-headers:options}
\textbf{--debug}
\begin{quote}

Put the script in verbose mode, useful for debugging.
\end{quote}

\textbf{--usage}
\begin{quote}

Prints a brief help message and exits.
\end{quote}

\textbf{--help}
\begin{quote}

Prints a more detailed help message and exits.
\end{quote}


\subsection{DESCRIPTION}
\label{doc-guide/parse-headers:description}
Convert a C header or source file (C\_FILE), into a ReStructured Text
included via ..parsed-literal block with cross-references for the
documentation files that describe the API. It accepts an optional
EXCEPTIONS\_FILE with describes what elements will be either ignored or
be pointed to a non-default reference.

The output is written at the (OUT\_FILE).

It is capable of identifying defines, functions, structs, typedefs,
enums and enum symbols and create cross-references for all of them.
It is also capable of distinguish \#define used for specifying a Linux
ioctl.

The EXCEPTIONS\_FILE contain two types of statements: \textbf{ignore} or \textbf{replace}.

The syntax for the ignore tag is:

ignore \textbf{type} \textbf{name}

The \textbf{ignore} means that it won't generate cross references for a
\textbf{name} symbol of type \textbf{type}.

The syntax for the replace tag is:

replace \textbf{type} \textbf{name} \textbf{new\_value}

The \textbf{replace} means that it will generate cross references for a
\textbf{name} symbol of type \textbf{type}, but, instead of using the default
replacement rule, it will use \textbf{new\_value}.

For both statements, \textbf{type} can be either one of the following:

\textbf{ioctl}
\begin{quote}

The ignore or replace statement will apply to ioctl definitions like:

\#define        VIDIOC\_DBG\_S\_REGISTER    \_IOW(`V', 79, struct v4l2\_dbg\_register)
\end{quote}

\textbf{define}
\begin{quote}

The ignore or replace statement will apply to any other \#define found
at C\_FILE.
\end{quote}

\textbf{typedef}
\begin{quote}

The ignore or replace statement will apply to typedef statements at C\_FILE.
\end{quote}

\textbf{struct}
\begin{quote}

The ignore or replace statement will apply to the name of struct statements
at C\_FILE.
\end{quote}

\textbf{enum}
\begin{quote}

The ignore or replace statement will apply to the name of enum statements
at C\_FILE.
\end{quote}

\textbf{symbol}
\begin{quote}

The ignore or replace statement will apply to the name of enum statements
at C\_FILE.

For replace statements, \textbf{new\_value} will automatically use :c:type:
references for \textbf{typedef}, \textbf{enum} and \textbf{struct} types. It will use :ref:
for \textbf{ioctl}, \textbf{define} and \textbf{symbol} types. The type of reference can
also be explicitly defined at the replace statement.
\end{quote}


\subsection{EXAMPLES}
\label{doc-guide/parse-headers:examples}
ignore define \_VIDEODEV2\_H

Ignore a \#define \_VIDEODEV2\_H at the C\_FILE.

ignore symbol PRIVATE

On a struct like:

enum foo \{ BAR1, BAR2, PRIVATE \};

It won't generate cross-references for \textbf{PRIVATE}.

replace symbol BAR1 :c:type:{}`foo{}`
replace symbol BAR2 :c:type:{}`foo{}`

On a struct like:

enum foo \{ BAR1, BAR2, PRIVATE \};

It will make the BAR1 and BAR2 enum symbols to cross reference the foo
symbol at the C domain.


\subsection{BUGS}
\label{doc-guide/parse-headers:bugs}
Report bugs to Mauro Carvalho Chehab \textless{}\href{mailto:mchehab@s-opensource.com}{mchehab@s-opensource.com}\textgreater{}


\subsection{COPYRIGHT}
\label{doc-guide/parse-headers:copyright}
Copyright (c) 2016 by Mauro Carvalho Chehab \textless{}\href{mailto:mchehab@s-opensource.com}{mchehab@s-opensource.com}\textgreater{}.

License GPLv2: GNU GPL version 2 \textless{}\href{http://gnu.org/licenses/gpl.html}{http://gnu.org/licenses/gpl.html}\textgreater{}.

This is free software: you are free to change and redistribute it.
There is NO WARRANTY, to the extent permitted by law.



\renewcommand{\indexname}{Index}
\printindex
\end{document}
