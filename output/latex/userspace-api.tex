% Generated by Sphinx.
\def\sphinxdocclass{report}
\documentclass[a4paper,8pt,english]{sphinxmanual}


\usepackage{cmap}
\usepackage[T1]{fontenc}
\usepackage{amsfonts}
\usepackage{babel}
\usepackage{times}
\usepackage[Bjarne]{fncychap}
\usepackage{longtable}
\usepackage{sphinx}
\usepackage{multirow}
\usepackage{eqparbox}


\addto\captionsenglish{\renewcommand{\figurename}{Fig. }}
\addto\captionsenglish{\renewcommand{\tablename}{Table }}
\SetupFloatingEnvironment{literal-block}{name=Listing }


	% Use some font with UTF-8 support with XeLaTeX
        \usepackage{fontspec}
        \setsansfont{DejaVu Serif}
        \setromanfont{DejaVu Sans}
        \setmonofont{DejaVu Sans Mono}

     \usepackage[margin=0.5in, top=1in, bottom=1in]{geometry}
        \usepackage{ifthen}

        % Put notes in color and let them be inside a table
	\definecolor{NoteColor}{RGB}{204,255,255}
	\definecolor{WarningColor}{RGB}{255,204,204}
	\definecolor{AttentionColor}{RGB}{255,255,204}
	\definecolor{ImportantColor}{RGB}{192,255,204}
	\definecolor{OtherColor}{RGB}{204,204,204}
        \newlength{\mynoticelength}
        \makeatletter\newenvironment{coloredbox}[1]{%
	   \setlength{\fboxrule}{1pt}
	   \setlength{\fboxsep}{7pt}
	   \setlength{\mynoticelength}{\linewidth}
	   \addtolength{\mynoticelength}{-2\fboxsep}
	   \addtolength{\mynoticelength}{-2\fboxrule}
           \begin{lrbox}{\@tempboxa}\begin{minipage}{\mynoticelength}}{\end{minipage}\end{lrbox}%
	   \ifthenelse%
	      {\equal{\py@noticetype}{note}}%
	      {\colorbox{NoteColor}{\usebox{\@tempboxa}}}%
	      {%
	         \ifthenelse%
	         {\equal{\py@noticetype}{warning}}%
	         {\colorbox{WarningColor}{\usebox{\@tempboxa}}}%
		 {%
	            \ifthenelse%
	            {\equal{\py@noticetype}{attention}}%
	            {\colorbox{AttentionColor}{\usebox{\@tempboxa}}}%
		    {%
	               \ifthenelse%
	               {\equal{\py@noticetype}{important}}%
	               {\colorbox{ImportantColor}{\usebox{\@tempboxa}}}%
	               {\colorbox{OtherColor}{\usebox{\@tempboxa}}}%
		    }%
		 }%
	      }%
        }\makeatother

        \makeatletter
        \renewenvironment{notice}[2]{%
          \def\py@noticetype{#1}
          \begin{coloredbox}{#1}
          \bf\it
          \par\strong{#2}
          \csname py@noticestart@#1\endcsname
        }
	{
          \csname py@noticeend@\py@noticetype\endcsname
          \end{coloredbox}
        }
	\makeatother

     

\title{The Linux kernel user-space API guide}
\date{March 08, 2018}
\release{4.16.0-rc4+}
\author{The kernel development community}
\newcommand{\sphinxlogo}{}
\renewcommand{\releasename}{Release}
\setcounter{tocdepth}{1}
\makeindex

\makeatletter
\def\PYG@reset{\let\PYG@it=\relax \let\PYG@bf=\relax%
    \let\PYG@ul=\relax \let\PYG@tc=\relax%
    \let\PYG@bc=\relax \let\PYG@ff=\relax}
\def\PYG@tok#1{\csname PYG@tok@#1\endcsname}
\def\PYG@toks#1+{\ifx\relax#1\empty\else%
    \PYG@tok{#1}\expandafter\PYG@toks\fi}
\def\PYG@do#1{\PYG@bc{\PYG@tc{\PYG@ul{%
    \PYG@it{\PYG@bf{\PYG@ff{#1}}}}}}}
\def\PYG#1#2{\PYG@reset\PYG@toks#1+\relax+\PYG@do{#2}}

\expandafter\def\csname PYG@tok@gd\endcsname{\def\PYG@tc##1{\textcolor[rgb]{0.63,0.00,0.00}{##1}}}
\expandafter\def\csname PYG@tok@gu\endcsname{\let\PYG@bf=\textbf\def\PYG@tc##1{\textcolor[rgb]{0.50,0.00,0.50}{##1}}}
\expandafter\def\csname PYG@tok@gt\endcsname{\def\PYG@tc##1{\textcolor[rgb]{0.00,0.27,0.87}{##1}}}
\expandafter\def\csname PYG@tok@gs\endcsname{\let\PYG@bf=\textbf}
\expandafter\def\csname PYG@tok@gr\endcsname{\def\PYG@tc##1{\textcolor[rgb]{1.00,0.00,0.00}{##1}}}
\expandafter\def\csname PYG@tok@cm\endcsname{\let\PYG@it=\textit\def\PYG@tc##1{\textcolor[rgb]{0.25,0.50,0.56}{##1}}}
\expandafter\def\csname PYG@tok@vg\endcsname{\def\PYG@tc##1{\textcolor[rgb]{0.73,0.38,0.84}{##1}}}
\expandafter\def\csname PYG@tok@vi\endcsname{\def\PYG@tc##1{\textcolor[rgb]{0.73,0.38,0.84}{##1}}}
\expandafter\def\csname PYG@tok@mh\endcsname{\def\PYG@tc##1{\textcolor[rgb]{0.13,0.50,0.31}{##1}}}
\expandafter\def\csname PYG@tok@cs\endcsname{\def\PYG@tc##1{\textcolor[rgb]{0.25,0.50,0.56}{##1}}\def\PYG@bc##1{\setlength{\fboxsep}{0pt}\colorbox[rgb]{1.00,0.94,0.94}{\strut ##1}}}
\expandafter\def\csname PYG@tok@ge\endcsname{\let\PYG@it=\textit}
\expandafter\def\csname PYG@tok@vc\endcsname{\def\PYG@tc##1{\textcolor[rgb]{0.73,0.38,0.84}{##1}}}
\expandafter\def\csname PYG@tok@il\endcsname{\def\PYG@tc##1{\textcolor[rgb]{0.13,0.50,0.31}{##1}}}
\expandafter\def\csname PYG@tok@go\endcsname{\def\PYG@tc##1{\textcolor[rgb]{0.20,0.20,0.20}{##1}}}
\expandafter\def\csname PYG@tok@cp\endcsname{\def\PYG@tc##1{\textcolor[rgb]{0.00,0.44,0.13}{##1}}}
\expandafter\def\csname PYG@tok@gi\endcsname{\def\PYG@tc##1{\textcolor[rgb]{0.00,0.63,0.00}{##1}}}
\expandafter\def\csname PYG@tok@gh\endcsname{\let\PYG@bf=\textbf\def\PYG@tc##1{\textcolor[rgb]{0.00,0.00,0.50}{##1}}}
\expandafter\def\csname PYG@tok@ni\endcsname{\let\PYG@bf=\textbf\def\PYG@tc##1{\textcolor[rgb]{0.84,0.33,0.22}{##1}}}
\expandafter\def\csname PYG@tok@nl\endcsname{\let\PYG@bf=\textbf\def\PYG@tc##1{\textcolor[rgb]{0.00,0.13,0.44}{##1}}}
\expandafter\def\csname PYG@tok@nn\endcsname{\let\PYG@bf=\textbf\def\PYG@tc##1{\textcolor[rgb]{0.05,0.52,0.71}{##1}}}
\expandafter\def\csname PYG@tok@no\endcsname{\def\PYG@tc##1{\textcolor[rgb]{0.38,0.68,0.84}{##1}}}
\expandafter\def\csname PYG@tok@na\endcsname{\def\PYG@tc##1{\textcolor[rgb]{0.25,0.44,0.63}{##1}}}
\expandafter\def\csname PYG@tok@nb\endcsname{\def\PYG@tc##1{\textcolor[rgb]{0.00,0.44,0.13}{##1}}}
\expandafter\def\csname PYG@tok@nc\endcsname{\let\PYG@bf=\textbf\def\PYG@tc##1{\textcolor[rgb]{0.05,0.52,0.71}{##1}}}
\expandafter\def\csname PYG@tok@nd\endcsname{\let\PYG@bf=\textbf\def\PYG@tc##1{\textcolor[rgb]{0.33,0.33,0.33}{##1}}}
\expandafter\def\csname PYG@tok@ne\endcsname{\def\PYG@tc##1{\textcolor[rgb]{0.00,0.44,0.13}{##1}}}
\expandafter\def\csname PYG@tok@nf\endcsname{\def\PYG@tc##1{\textcolor[rgb]{0.02,0.16,0.49}{##1}}}
\expandafter\def\csname PYG@tok@si\endcsname{\let\PYG@it=\textit\def\PYG@tc##1{\textcolor[rgb]{0.44,0.63,0.82}{##1}}}
\expandafter\def\csname PYG@tok@s2\endcsname{\def\PYG@tc##1{\textcolor[rgb]{0.25,0.44,0.63}{##1}}}
\expandafter\def\csname PYG@tok@nt\endcsname{\let\PYG@bf=\textbf\def\PYG@tc##1{\textcolor[rgb]{0.02,0.16,0.45}{##1}}}
\expandafter\def\csname PYG@tok@nv\endcsname{\def\PYG@tc##1{\textcolor[rgb]{0.73,0.38,0.84}{##1}}}
\expandafter\def\csname PYG@tok@s1\endcsname{\def\PYG@tc##1{\textcolor[rgb]{0.25,0.44,0.63}{##1}}}
\expandafter\def\csname PYG@tok@ch\endcsname{\let\PYG@it=\textit\def\PYG@tc##1{\textcolor[rgb]{0.25,0.50,0.56}{##1}}}
\expandafter\def\csname PYG@tok@m\endcsname{\def\PYG@tc##1{\textcolor[rgb]{0.13,0.50,0.31}{##1}}}
\expandafter\def\csname PYG@tok@gp\endcsname{\let\PYG@bf=\textbf\def\PYG@tc##1{\textcolor[rgb]{0.78,0.36,0.04}{##1}}}
\expandafter\def\csname PYG@tok@sh\endcsname{\def\PYG@tc##1{\textcolor[rgb]{0.25,0.44,0.63}{##1}}}
\expandafter\def\csname PYG@tok@ow\endcsname{\let\PYG@bf=\textbf\def\PYG@tc##1{\textcolor[rgb]{0.00,0.44,0.13}{##1}}}
\expandafter\def\csname PYG@tok@sx\endcsname{\def\PYG@tc##1{\textcolor[rgb]{0.78,0.36,0.04}{##1}}}
\expandafter\def\csname PYG@tok@bp\endcsname{\def\PYG@tc##1{\textcolor[rgb]{0.00,0.44,0.13}{##1}}}
\expandafter\def\csname PYG@tok@c1\endcsname{\let\PYG@it=\textit\def\PYG@tc##1{\textcolor[rgb]{0.25,0.50,0.56}{##1}}}
\expandafter\def\csname PYG@tok@o\endcsname{\def\PYG@tc##1{\textcolor[rgb]{0.40,0.40,0.40}{##1}}}
\expandafter\def\csname PYG@tok@kc\endcsname{\let\PYG@bf=\textbf\def\PYG@tc##1{\textcolor[rgb]{0.00,0.44,0.13}{##1}}}
\expandafter\def\csname PYG@tok@c\endcsname{\let\PYG@it=\textit\def\PYG@tc##1{\textcolor[rgb]{0.25,0.50,0.56}{##1}}}
\expandafter\def\csname PYG@tok@mf\endcsname{\def\PYG@tc##1{\textcolor[rgb]{0.13,0.50,0.31}{##1}}}
\expandafter\def\csname PYG@tok@err\endcsname{\def\PYG@bc##1{\setlength{\fboxsep}{0pt}\fcolorbox[rgb]{1.00,0.00,0.00}{1,1,1}{\strut ##1}}}
\expandafter\def\csname PYG@tok@mb\endcsname{\def\PYG@tc##1{\textcolor[rgb]{0.13,0.50,0.31}{##1}}}
\expandafter\def\csname PYG@tok@ss\endcsname{\def\PYG@tc##1{\textcolor[rgb]{0.32,0.47,0.09}{##1}}}
\expandafter\def\csname PYG@tok@sr\endcsname{\def\PYG@tc##1{\textcolor[rgb]{0.14,0.33,0.53}{##1}}}
\expandafter\def\csname PYG@tok@mo\endcsname{\def\PYG@tc##1{\textcolor[rgb]{0.13,0.50,0.31}{##1}}}
\expandafter\def\csname PYG@tok@kd\endcsname{\let\PYG@bf=\textbf\def\PYG@tc##1{\textcolor[rgb]{0.00,0.44,0.13}{##1}}}
\expandafter\def\csname PYG@tok@mi\endcsname{\def\PYG@tc##1{\textcolor[rgb]{0.13,0.50,0.31}{##1}}}
\expandafter\def\csname PYG@tok@kn\endcsname{\let\PYG@bf=\textbf\def\PYG@tc##1{\textcolor[rgb]{0.00,0.44,0.13}{##1}}}
\expandafter\def\csname PYG@tok@cpf\endcsname{\let\PYG@it=\textit\def\PYG@tc##1{\textcolor[rgb]{0.25,0.50,0.56}{##1}}}
\expandafter\def\csname PYG@tok@kr\endcsname{\let\PYG@bf=\textbf\def\PYG@tc##1{\textcolor[rgb]{0.00,0.44,0.13}{##1}}}
\expandafter\def\csname PYG@tok@s\endcsname{\def\PYG@tc##1{\textcolor[rgb]{0.25,0.44,0.63}{##1}}}
\expandafter\def\csname PYG@tok@kp\endcsname{\def\PYG@tc##1{\textcolor[rgb]{0.00,0.44,0.13}{##1}}}
\expandafter\def\csname PYG@tok@w\endcsname{\def\PYG@tc##1{\textcolor[rgb]{0.73,0.73,0.73}{##1}}}
\expandafter\def\csname PYG@tok@kt\endcsname{\def\PYG@tc##1{\textcolor[rgb]{0.56,0.13,0.00}{##1}}}
\expandafter\def\csname PYG@tok@sc\endcsname{\def\PYG@tc##1{\textcolor[rgb]{0.25,0.44,0.63}{##1}}}
\expandafter\def\csname PYG@tok@sb\endcsname{\def\PYG@tc##1{\textcolor[rgb]{0.25,0.44,0.63}{##1}}}
\expandafter\def\csname PYG@tok@k\endcsname{\let\PYG@bf=\textbf\def\PYG@tc##1{\textcolor[rgb]{0.00,0.44,0.13}{##1}}}
\expandafter\def\csname PYG@tok@se\endcsname{\let\PYG@bf=\textbf\def\PYG@tc##1{\textcolor[rgb]{0.25,0.44,0.63}{##1}}}
\expandafter\def\csname PYG@tok@sd\endcsname{\let\PYG@it=\textit\def\PYG@tc##1{\textcolor[rgb]{0.25,0.44,0.63}{##1}}}

\def\PYGZbs{\char`\\}
\def\PYGZus{\char`\_}
\def\PYGZob{\char`\{}
\def\PYGZcb{\char`\}}
\def\PYGZca{\char`\^}
\def\PYGZam{\char`\&}
\def\PYGZlt{\char`\<}
\def\PYGZgt{\char`\>}
\def\PYGZsh{\char`\#}
\def\PYGZpc{\char`\%}
\def\PYGZdl{\char`\$}
\def\PYGZhy{\char`\-}
\def\PYGZsq{\char`\'}
\def\PYGZdq{\char`\"}
\def\PYGZti{\char`\~}
% for compatibility with earlier versions
\def\PYGZat{@}
\def\PYGZlb{[}
\def\PYGZrb{]}
\makeatother

\renewcommand\PYGZsq{\textquotesingle}

\begin{document}

\maketitle
\tableofcontents
\phantomsection\label{userspace-api/index::doc}


While much of the kernel's user-space API is documented elsewhere
(particularly in the \href{https://www.kernel.org/doc/man-pages/}{man-pages} project), some user-space information can
also be found in the kernel tree itself.  This manual is intended to be the
place where this information is gathered.

Table of contents


\chapter{No New Privileges Flag}
\label{userspace-api/no_new_privs::doc}\label{userspace-api/no_new_privs:no-new-privileges-flag}\label{userspace-api/no_new_privs:the-linux-kernel-user-space-api-guide}
The execve system call can grant a newly-started program privileges that
its parent did not have.  The most obvious examples are setuid/setgid
programs and file capabilities.  To prevent the parent program from
gaining these privileges as well, the kernel and user code must be
careful to prevent the parent from doing anything that could subvert the
child.  For example:
\begin{itemize}
\item {} 
The dynamic loader handles \code{LD\_*} environment variables differently if
a program is setuid.

\item {} 
chroot is disallowed to unprivileged processes, since it would allow
\code{/etc/passwd} to be replaced from the point of view of a process that
inherited chroot.

\item {} 
The exec code has special handling for ptrace.

\end{itemize}

These are all ad-hoc fixes.  The \code{no\_new\_privs} bit (since Linux 3.5) is a
new, generic mechanism to make it safe for a process to modify its
execution environment in a manner that persists across execve.  Any task
can set \code{no\_new\_privs}.  Once the bit is set, it is inherited across fork,
clone, and execve and cannot be unset.  With \code{no\_new\_privs} set, \code{execve()}
promises not to grant the privilege to do anything that could not have
been done without the execve call.  For example, the setuid and setgid
bits will no longer change the uid or gid; file capabilities will not
add to the permitted set, and LSMs will not relax constraints after
execve.

To set \code{no\_new\_privs}, use:

\begin{Verbatim}[commandchars=\\\{\}]
prctl(PR\PYGZus{}SET\PYGZus{}NO\PYGZus{}NEW\PYGZus{}PRIVS, 1, 0, 0, 0);
\end{Verbatim}

Be careful, though: LSMs might also not tighten constraints on exec
in \code{no\_new\_privs} mode.  (This means that setting up a general-purpose
service launcher to set \code{no\_new\_privs} before execing daemons may
interfere with LSM-based sandboxing.)

Note that \code{no\_new\_privs} does not prevent privilege changes that do not
involve \code{execve()}.  An appropriately privileged task can still call
\code{setuid(2)} and receive SCM\_RIGHTS datagrams.

There are two main use cases for \code{no\_new\_privs} so far:
\begin{itemize}
\item {} 
Filters installed for the seccomp mode 2 sandbox persist across
execve and can change the behavior of newly-executed programs.
Unprivileged users are therefore only allowed to install such filters
if \code{no\_new\_privs} is set.

\item {} 
By itself, \code{no\_new\_privs} can be used to reduce the attack surface
available to an unprivileged user.  If everything running with a
given uid has \code{no\_new\_privs} set, then that uid will be unable to
escalate its privileges by directly attacking setuid, setgid, and
fcap-using binaries; it will need to compromise something without the
\code{no\_new\_privs} bit set first.

\end{itemize}

In the future, other potentially dangerous kernel features could become
available to unprivileged tasks if \code{no\_new\_privs} is set.  In principle,
several options to \code{unshare(2)} and \code{clone(2)} would be safe when
\code{no\_new\_privs} is set, and \code{no\_new\_privs} + \code{chroot} is considerable less
dangerous than chroot by itself.


\chapter{Seccomp BPF (SECure COMPuting with filters)}
\label{userspace-api/seccomp_filter::doc}\label{userspace-api/seccomp_filter:seccomp-bpf-secure-computing-with-filters}

\section{Introduction}
\label{userspace-api/seccomp_filter:introduction}
A large number of system calls are exposed to every userland process
with many of them going unused for the entire lifetime of the process.
As system calls change and mature, bugs are found and eradicated.  A
certain subset of userland applications benefit by having a reduced set
of available system calls.  The resulting set reduces the total kernel
surface exposed to the application.  System call filtering is meant for
use with those applications.

Seccomp filtering provides a means for a process to specify a filter for
incoming system calls.  The filter is expressed as a Berkeley Packet
Filter (BPF) program, as with socket filters, except that the data
operated on is related to the system call being made: system call
number and the system call arguments.  This allows for expressive
filtering of system calls using a filter program language with a long
history of being exposed to userland and a straightforward data set.

Additionally, BPF makes it impossible for users of seccomp to fall prey
to time-of-check-time-of-use (TOCTOU) attacks that are common in system
call interposition frameworks.  BPF programs may not dereference
pointers which constrains all filters to solely evaluating the system
call arguments directly.


\section{What it isn't}
\label{userspace-api/seccomp_filter:what-it-isn-t}
System call filtering isn't a sandbox.  It provides a clearly defined
mechanism for minimizing the exposed kernel surface.  It is meant to be
a tool for sandbox developers to use.  Beyond that, policy for logical
behavior and information flow should be managed with a combination of
other system hardening techniques and, potentially, an LSM of your
choosing.  Expressive, dynamic filters provide further options down this
path (avoiding pathological sizes or selecting which of the multiplexed
system calls in socketcall() is allowed, for instance) which could be
construed, incorrectly, as a more complete sandboxing solution.


\section{Usage}
\label{userspace-api/seccomp_filter:usage}
An additional seccomp mode is added and is enabled using the same
prctl(2) call as the strict seccomp.  If the architecture has
\code{CONFIG\_HAVE\_ARCH\_SECCOMP\_FILTER}, then filters may be added as below:
\begin{description}
\item[{\code{PR\_SET\_SECCOMP}:}] \leavevmode
Now takes an additional argument which specifies a new filter
using a BPF program.
The BPF program will be executed over struct seccomp\_data
reflecting the system call number, arguments, and other
metadata.  The BPF program must then return one of the
acceptable values to inform the kernel which action should be
taken.

Usage:

\begin{Verbatim}[commandchars=\\\{\}]
prctl(PR\PYGZus{}SET\PYGZus{}SECCOMP, SECCOMP\PYGZus{}MODE\PYGZus{}FILTER, prog);
\end{Verbatim}

The `prog' argument is a pointer to a struct sock\_fprog which
will contain the filter program.  If the program is invalid, the
call will return -1 and set errno to \code{EINVAL}.

If \code{fork}/\code{clone} and \code{execve} are allowed by @prog, any child
processes will be constrained to the same filters and system
call ABI as the parent.

Prior to use, the task must call \code{prctl(PR\_SET\_NO\_NEW\_PRIVS, 1)} or
run with \code{CAP\_SYS\_ADMIN} privileges in its namespace.  If these are not
true, \code{-EACCES} will be returned.  This requirement ensures that filter
programs cannot be applied to child processes with greater privileges
than the task that installed them.

Additionally, if \code{prctl(2)} is allowed by the attached filter,
additional filters may be layered on which will increase evaluation
time, but allow for further decreasing the attack surface during
execution of a process.

\end{description}

The above call returns 0 on success and non-zero on error.


\section{Return values}
\label{userspace-api/seccomp_filter:return-values}
A seccomp filter may return any of the following values. If multiple
filters exist, the return value for the evaluation of a given system
call will always use the highest precedent value. (For example,
\code{SECCOMP\_RET\_KILL\_PROCESS} will always take precedence.)

In precedence order, they are:
\begin{description}
\item[{\code{SECCOMP\_RET\_KILL\_PROCESS}:}] \leavevmode
Results in the entire process exiting immediately without executing
the system call.  The exit status of the task (\code{status \& 0x7f})
will be \code{SIGSYS}, not \code{SIGKILL}.

\item[{\code{SECCOMP\_RET\_KILL\_THREAD}:}] \leavevmode
Results in the task exiting immediately without executing the
system call.  The exit status of the task (\code{status \& 0x7f}) will
be \code{SIGSYS}, not \code{SIGKILL}.

\item[{\code{SECCOMP\_RET\_TRAP}:}] \leavevmode
Results in the kernel sending a \code{SIGSYS} signal to the triggering
task without executing the system call. \code{siginfo-\textgreater{}si\_call\_addr}
will show the address of the system call instruction, and
\code{siginfo-\textgreater{}si\_syscall} and \code{siginfo-\textgreater{}si\_arch} will indicate which
syscall was attempted.  The program counter will be as though
the syscall happened (i.e. it will not point to the syscall
instruction).  The return value register will contain an arch-
dependent value -- if resuming execution, set it to something
sensible.  (The architecture dependency is because replacing
it with \code{-ENOSYS} could overwrite some useful information.)

The \code{SECCOMP\_RET\_DATA} portion of the return value will be passed
as \code{si\_errno}.

\code{SIGSYS} triggered by seccomp will have a si\_code of \code{SYS\_SECCOMP}.

\item[{\code{SECCOMP\_RET\_ERRNO}:}] \leavevmode
Results in the lower 16-bits of the return value being passed
to userland as the errno without executing the system call.

\item[{\code{SECCOMP\_RET\_TRACE}:}] \leavevmode
When returned, this value will cause the kernel to attempt to
notify a \code{ptrace()}-based tracer prior to executing the system
call.  If there is no tracer present, \code{-ENOSYS} is returned to
userland and the system call is not executed.

A tracer will be notified if it requests \code{PTRACE\_O\_TRACESECCOM{}`{}`P
using {}`{}`ptrace(PTRACE\_SETOPTIONS)}.  The tracer will be notified
of a \code{PTRACE\_EVENT\_SECCOMP} and the \code{SECCOMP\_RET\_DATA} portion of
the BPF program return value will be available to the tracer
via \code{PTRACE\_GETEVENTMSG}.

The tracer can skip the system call by changing the syscall number
to -1.  Alternatively, the tracer can change the system call
requested by changing the system call to a valid syscall number.  If
the tracer asks to skip the system call, then the system call will
appear to return the value that the tracer puts in the return value
register.

The seccomp check will not be run again after the tracer is
notified.  (This means that seccomp-based sandboxes MUST NOT
allow use of ptrace, even of other sandboxed processes, without
extreme care; ptracers can use this mechanism to escape.)

\item[{\code{SECCOMP\_RET\_LOG}:}] \leavevmode
Results in the system call being executed after it is logged. This
should be used by application developers to learn which syscalls their
application needs without having to iterate through multiple test and
development cycles to build the list.

This action will only be logged if ``log'' is present in the
actions\_logged sysctl string.

\item[{\code{SECCOMP\_RET\_ALLOW}:}] \leavevmode
Results in the system call being executed.

\end{description}

If multiple filters exist, the return value for the evaluation of a
given system call will always use the highest precedent value.

Precedence is only determined using the \code{SECCOMP\_RET\_ACTION} mask.  When
multiple filters return values of the same precedence, only the
\code{SECCOMP\_RET\_DATA} from the most recently installed filter will be
returned.


\section{Pitfalls}
\label{userspace-api/seccomp_filter:pitfalls}
The biggest pitfall to avoid during use is filtering on system call
number without checking the architecture value.  Why?  On any
architecture that supports multiple system call invocation conventions,
the system call numbers may vary based on the specific invocation.  If
the numbers in the different calling conventions overlap, then checks in
the filters may be abused.  Always check the arch value!


\section{Example}
\label{userspace-api/seccomp_filter:example}
The \code{samples/seccomp/} directory contains both an x86-specific example
and a more generic example of a higher level macro interface for BPF
program generation.


\section{Sysctls}
\label{userspace-api/seccomp_filter:sysctls}
Seccomp's sysctl files can be found in the \code{/proc/sys/kernel/seccomp/}
directory. Here's a description of each file in that directory:
\begin{description}
\item[{\code{actions\_avail}:}] \leavevmode
A read-only ordered list of seccomp return values (refer to the
\code{SECCOMP\_RET\_*} macros above) in string form. The ordering, from
left-to-right, is the least permissive return value to the most
permissive return value.

The list represents the set of seccomp return values supported
by the kernel. A userspace program may use this list to
determine if the actions found in the \code{seccomp.h}, when the
program was built, differs from the set of actions actually
supported in the current running kernel.

\item[{\code{actions\_logged}:}] \leavevmode
A read-write ordered list of seccomp return values (refer to the
\code{SECCOMP\_RET\_*} macros above) that are allowed to be logged. Writes
to the file do not need to be in ordered form but reads from the file
will be ordered in the same way as the actions\_avail sysctl.

It is important to note that the value of \code{actions\_logged} does not
prevent certain actions from being logged when the audit subsystem is
configured to audit a task. If the action is not found in
\code{actions\_logged} list, the final decision on whether to audit the
action for that task is ultimately left up to the audit subsystem to
decide for all seccomp return values other than \code{SECCOMP\_RET\_ALLOW}.

The \code{allow} string is not accepted in the \code{actions\_logged} sysctl
as it is not possible to log \code{SECCOMP\_RET\_ALLOW} actions. Attempting
to write \code{allow} to the sysctl will result in an EINVAL being
returned.

\end{description}


\section{Adding architecture support}
\label{userspace-api/seccomp_filter:adding-architecture-support}
See \code{arch/Kconfig} for the authoritative requirements.  In general, if an
architecture supports both ptrace\_event and seccomp, it will be able to
support seccomp filter with minor fixup: \code{SIGSYS} support and seccomp return
value checking.  Then it must just add \code{CONFIG\_HAVE\_ARCH\_SECCOMP\_FILTER}
to its arch-specific Kconfig.


\section{Caveats}
\label{userspace-api/seccomp_filter:caveats}
The vDSO can cause some system calls to run entirely in userspace,
leading to surprises when you run programs on different machines that
fall back to real syscalls.  To minimize these surprises on x86, make
sure you test with
\code{/sys/devices/system/clocksource/clocksource0/current\_clocksource} set to
something like \code{acpi\_pm}.

On x86-64, vsyscall emulation is enabled by default.  (vsyscalls are
legacy variants on vDSO calls.)  Currently, emulated vsyscalls will
honor seccomp, with a few oddities:
\begin{itemize}
\item {} 
A return value of \code{SECCOMP\_RET\_TRAP} will set a \code{si\_call\_addr} pointing to
the vsyscall entry for the given call and not the address after the
`syscall' instruction.  Any code which wants to restart the call
should be aware that (a) a ret instruction has been emulated and (b)
trying to resume the syscall will again trigger the standard vsyscall
emulation security checks, making resuming the syscall mostly
pointless.

\item {} 
A return value of \code{SECCOMP\_RET\_TRACE} will signal the tracer as usual,
but the syscall may not be changed to another system call using the
orig\_rax register. It may only be changed to -1 order to skip the
currently emulated call. Any other change MAY terminate the process.
The rip value seen by the tracer will be the syscall entry address;
this is different from normal behavior.  The tracer MUST NOT modify
rip or rsp.  (Do not rely on other changes terminating the process.
They might work.  For example, on some kernels, choosing a syscall
that only exists in future kernels will be correctly emulated (by
returning \code{-ENOSYS}).

\end{itemize}

To detect this quirky behavior, check for \code{addr \& \textasciitilde{}0x0C00 ==
0xFFFFFFFFFF600000}.  (For \code{SECCOMP\_RET\_TRACE}, use rip.  For
\code{SECCOMP\_RET\_TRAP}, use \code{siginfo-\textgreater{}si\_call\_addr}.)  Do not check any other
condition: future kernels may improve vsyscall emulation and current
kernels in vsyscall=native mode will behave differently, but the
instructions at \code{0xF...F600\{0,4,8,C\}00} will not be system calls in these
cases.

Note that modern systems are unlikely to use vsyscalls at all -- they
are a legacy feature and they are considerably slower than standard
syscalls.  New code will use the vDSO, and vDSO-issued system calls
are indistinguishable from normal system calls.


\chapter{unshare system call}
\label{userspace-api/unshare:unshare-system-call}\label{userspace-api/unshare::doc}
This document describes the new system call, unshare(). The document
provides an overview of the feature, why it is needed, how it can
be used, its interface specification, design, implementation and
how it can be tested.


\section{Change Log}
\label{userspace-api/unshare:change-log}
version 0.1  Initial document, Janak Desai (\href{mailto:janak@us.ibm.com}{janak@us.ibm.com}), Jan 11, 2006


\section{Contents}
\label{userspace-api/unshare:contents}\begin{enumerate}
\item {} 
Overview

\item {} 
Benefits

\item {} 
Cost

\item {} 
Requirements

\item {} 
Functional Specification

\item {} 
High Level Design

\item {} 
Low Level Design

\item {} 
Test Specification

\item {} 
Future Work

\end{enumerate}


\section{1) Overview}
\label{userspace-api/unshare:overview}
Most legacy operating system kernels support an abstraction of threads
as multiple execution contexts within a process. These kernels provide
special resources and mechanisms to maintain these ``threads''. The Linux
kernel, in a clever and simple manner, does not make distinction
between processes and ``threads''. The kernel allows processes to share
resources and thus they can achieve legacy ``threads'' behavior without
requiring additional data structures and mechanisms in the kernel. The
power of implementing threads in this manner comes not only from
its simplicity but also from allowing application programmers to work
outside the confinement of all-or-nothing shared resources of legacy
threads. On Linux, at the time of thread creation using the clone system
call, applications can selectively choose which resources to share
between threads.

unshare() system call adds a primitive to the Linux thread model that
allows threads to selectively `unshare' any resources that were being
shared at the time of their creation. unshare() was conceptualized by
Al Viro in the August of 2000, on the Linux-Kernel mailing list, as part
of the discussion on POSIX threads on Linux.  unshare() augments the
usefulness of Linux threads for applications that would like to control
shared resources without creating a new process. unshare() is a natural
addition to the set of available primitives on Linux that implement
the concept of process/thread as a virtual machine.


\section{2) Benefits}
\label{userspace-api/unshare:benefits}
unshare() would be useful to large application frameworks such as PAM
where creating a new process to control sharing/unsharing of process
resources is not possible. Since namespaces are shared by default
when creating a new process using fork or clone, unshare() can benefit
even non-threaded applications if they have a need to disassociate
from default shared namespace. The following lists two use-cases
where unshare() can be used.


\subsection{2.1 Per-security context namespaces}
\label{userspace-api/unshare:per-security-context-namespaces}
unshare() can be used to implement polyinstantiated directories using
the kernel's per-process namespace mechanism. Polyinstantiated directories,
such as per-user and/or per-security context instance of /tmp, /var/tmp or
per-security context instance of a user's home directory, isolate user
processes when working with these directories. Using unshare(), a PAM
module can easily setup a private namespace for a user at login.
Polyinstantiated directories are required for Common Criteria certification
with Labeled System Protection Profile, however, with the availability
of shared-tree feature in the Linux kernel, even regular Linux systems
can benefit from setting up private namespaces at login and
polyinstantiating /tmp, /var/tmp and other directories deemed
appropriate by system administrators.


\subsection{2.2 unsharing of virtual memory and/or open files}
\label{userspace-api/unshare:unsharing-of-virtual-memory-and-or-open-files}
Consider a client/server application where the server is processing
client requests by creating processes that share resources such as
virtual memory and open files. Without unshare(), the server has to
decide what needs to be shared at the time of creating the process
which services the request. unshare() allows the server an ability to
disassociate parts of the context during the servicing of the
request. For large and complex middleware application frameworks, this
ability to unshare() after the process was created can be very
useful.


\section{3) Cost}
\label{userspace-api/unshare:cost}
In order to not duplicate code and to handle the fact that unshare()
works on an active task (as opposed to clone/fork working on a newly
allocated inactive task) unshare() had to make minor reorganizational
changes to copy\_* functions utilized by clone/fork system call.
There is a cost associated with altering existing, well tested and
stable code to implement a new feature that may not get exercised
extensively in the beginning. However, with proper design and code
review of the changes and creation of an unshare() test for the LTP
the benefits of this new feature can exceed its cost.


\section{4) Requirements}
\label{userspace-api/unshare:requirements}
unshare() reverses sharing that was done using clone(2) system call,
so unshare() should have a similar interface as clone(2). That is,
since flags in clone(int flags, void *stack) specifies what should
be shared, similar flags in unshare(int flags) should specify
what should be unshared. Unfortunately, this may appear to invert
the meaning of the flags from the way they are used in clone(2).
However, there was no easy solution that was less confusing and that
allowed incremental context unsharing in future without an ABI change.

unshare() interface should accommodate possible future addition of
new context flags without requiring a rebuild of old applications.
If and when new context flags are added, unshare() design should allow
incremental unsharing of those resources on an as needed basis.


\section{5) Functional Specification}
\label{userspace-api/unshare:functional-specification}\begin{description}
\item[{NAME}] \leavevmode
unshare - disassociate parts of the process execution context

\item[{SYNOPSIS}] \leavevmode
\#include \textless{}sched.h\textgreater{}

int unshare(int flags);

\item[{DESCRIPTION}] \leavevmode
unshare() allows a process to disassociate parts of its execution
context that are currently being shared with other processes. Part
of execution context, such as the namespace, is shared by default
when a new process is created using fork(2), while other parts,
such as the virtual memory, open file descriptors, etc, may be
shared by explicit request to share them when creating a process
using clone(2).

The main use of unshare() is to allow a process to control its
shared execution context without creating a new process.

The flags argument specifies one or bitwise-or'ed of several of
the following constants.
\begin{description}
\item[{CLONE\_FS}] \leavevmode
If CLONE\_FS is set, file system information of the caller
is disassociated from the shared file system information.

\item[{CLONE\_FILES}] \leavevmode
If CLONE\_FILES is set, the file descriptor table of the
caller is disassociated from the shared file descriptor
table.

\item[{CLONE\_NEWNS}] \leavevmode
If CLONE\_NEWNS is set, the namespace of the caller is
disassociated from the shared namespace.

\item[{CLONE\_VM}] \leavevmode
If CLONE\_VM is set, the virtual memory of the caller is
disassociated from the shared virtual memory.

\end{description}

\item[{RETURN VALUE}] \leavevmode
On success, zero returned. On failure, -1 is returned and errno is

\item[{ERRORS}] \leavevmode\begin{description}
\item[{EPERM   CLONE\_NEWNS was specified by a non-root process (process}] \leavevmode
without CAP\_SYS\_ADMIN).

\item[{ENOMEM  Cannot allocate sufficient memory to copy parts of caller's}] \leavevmode
context that need to be unshared.

\end{description}

EINVAL  Invalid flag was specified as an argument.

\item[{CONFORMING TO}] \leavevmode
The unshare() call is Linux-specific and  should  not be used
in programs intended to be portable.

\item[{SEE ALSO}] \leavevmode
clone(2), fork(2)

\end{description}


\section{6) High Level Design}
\label{userspace-api/unshare:high-level-design}
Depending on the flags argument, the unshare() system call allocates
appropriate process context structures, populates it with values from
the current shared version, associates newly duplicated structures
with the current task structure and releases corresponding shared
versions. Helper functions of clone (copy\_*) could not be used
directly by unshare() because of the following two reasons.
\begin{enumerate}
\item {} 
clone operates on a newly allocated not-yet-active task
structure, where as unshare() operates on the current active
task. Therefore unshare() has to take appropriate task\_lock()
before associating newly duplicated context structures

\item {} 
unshare() has to allocate and duplicate all context structures
that are being unshared, before associating them with the
current task and releasing older shared structures. Failure
do so will create race conditions and/or oops when trying
to backout due to an error. Consider the case of unsharing
both virtual memory and namespace. After successfully unsharing
vm, if the system call encounters an error while allocating
new namespace structure, the error return code will have to
reverse the unsharing of vm. As part of the reversal the
system call will have to go back to older, shared, vm
structure, which may not exist anymore.

\end{enumerate}

Therefore code from copy\_* functions that allocated and duplicated
current context structure was moved into new dup\_* functions. Now,
copy\_* functions call dup\_* functions to allocate and duplicate
appropriate context structures and then associate them with the
task structure that is being constructed. unshare() system call on
the other hand performs the following:
\begin{enumerate}
\item {} 
Check flags to force missing, but implied, flags

\item {} 
For each context structure, call the corresponding unshare()
helper function to allocate and duplicate a new context
structure, if the appropriate bit is set in the flags argument.

\item {} 
If there is no error in allocation and duplication and there
are new context structures then lock the current task structure,
associate new context structures with the current task structure,
and release the lock on the current task structure.

\item {} 
Appropriately release older, shared, context structures.

\end{enumerate}


\section{7) Low Level Design}
\label{userspace-api/unshare:low-level-design}
Implementation of unshare() can be grouped in the following 4 different
items:
\begin{enumerate}
\item {} 
Reorganization of existing copy\_* functions

\item {} 
unshare() system call service function

\item {} 
unshare() helper functions for each different process context

\item {} 
Registration of system call number for different architectures

\end{enumerate}


\subsection{7.1) Reorganization of copy\_* functions}
\label{userspace-api/unshare:reorganization-of-copy-functions}
Each copy function such as copy\_mm, copy\_namespace, copy\_files,
etc, had roughly two components. The first component allocated
and duplicated the appropriate structure and the second component
linked it to the task structure passed in as an argument to the copy
function. The first component was split into its own function.
These dup\_* functions allocated and duplicated the appropriate
context structure. The reorganized copy\_* functions invoked
their corresponding dup\_* functions and then linked the newly
duplicated structures to the task structure with which the
copy function was called.


\subsection{7.2) unshare() system call service function}
\label{userspace-api/unshare:unshare-system-call-service-function}\begin{itemize}
\item {} 
Check flags
Force implied flags. If CLONE\_THREAD is set force CLONE\_VM.
If CLONE\_VM is set, force CLONE\_SIGHAND. If CLONE\_SIGHAND is
set and signals are also being shared, force CLONE\_THREAD. If
CLONE\_NEWNS is set, force CLONE\_FS.

\item {} 
For each context flag, invoke the corresponding unshare\_*
helper routine with flags passed into the system call and a
reference to pointer pointing the new unshared structure

\item {} 
If any new structures are created by unshare\_* helper
functions, take the task\_lock() on the current task,
modify appropriate context pointers, and release the
task lock.

\item {} 
For all newly unshared structures, release the corresponding
older, shared, structures.

\end{itemize}


\subsection{7.3) unshare\_* helper functions}
\label{userspace-api/unshare:unshare-helper-functions}
For unshare\_* helpers corresponding to CLONE\_SYSVSEM, CLONE\_SIGHAND,
and CLONE\_THREAD, return -EINVAL since they are not implemented yet.
For others, check the flag value to see if the unsharing is
required for that structure. If it is, invoke the corresponding
dup\_* function to allocate and duplicate the structure and return
a pointer to it.


\subsection{7.4) Finally}
\label{userspace-api/unshare:finally}
Appropriately modify architecture specific code to register the
new system call.


\section{8) Test Specification}
\label{userspace-api/unshare:test-specification}
The test for unshare() should test the following:
\begin{enumerate}
\item {} 
Valid flags: Test to check that clone flags for signal and
signal handlers, for which unsharing is not implemented
yet, return -EINVAL.

\item {} 
Missing/implied flags: Test to make sure that if unsharing
namespace without specifying unsharing of filesystem, correctly
unshares both namespace and filesystem information.

\item {} 
For each of the four (namespace, filesystem, files and vm)
supported unsharing, verify that the system call correctly
unshares the appropriate structure. Verify that unsharing
them individually as well as in combination with each
other works as expected.

\item {} 
Concurrent execution: Use shared memory segments and futex on
an address in the shm segment to synchronize execution of
about 10 threads. Have a couple of threads execute execve,
a couple \_exit and the rest unshare with different combination
of flags. Verify that unsharing is performed as expected and
that there are no oops or hangs.

\end{enumerate}


\section{9) Future Work}
\label{userspace-api/unshare:future-work}
The current implementation of unshare() does not allow unsharing of
signals and signal handlers. Signals are complex to begin with and
to unshare signals and/or signal handlers of a currently running
process is even more complex. If in the future there is a specific
need to allow unsharing of signals and/or signal handlers, it can
be incrementally added to unshare() without affecting legacy
applications using unshare().



\renewcommand{\indexname}{Index}
\printindex
\end{document}
